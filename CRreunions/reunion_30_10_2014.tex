\documentclass[10pt,a4paper]{article}
\usepackage[utf8]{inputenc}
\usepackage[T1]{fontenc}
\usepackage[french]{babel}
\usepackage{enumitem}
%\frenchbsetup{StandardLists=true}
\usepackage{caption}
\usepackage{graphicx}
\usepackage{amsmath}
\usepackage{subfigure}
\usepackage{appendix}
\usepackage{listings}
\usepackage{multicol}
\usepackage{moreverb}
\usepackage{epsfig}
\usepackage{textcomp}


    \usepackage{xcolor}
      %%%%%%%%%%%%%%%%  inclure la source %%%%%%%%%%%%%%%%%%%%
  \newcommand*\styleC{\fontsize{9}{10pt}\usefont{T1}{cmtt}{m}{n}\selectfont }
  \newcommand*\styleD{\fontsize{9}{10pt}\usefont{T1}{cmtt}{m}{n}\selectfont }

      \makeatletter
      % on fixe le langage utilisé
      \lstset{language=matlab, breaklines=true}
      \edef\Motscle{emph={\lst@keywords}}
      \expandafter\lstset\expandafter{%
       \Motscle}
      \makeatother


      \definecolor{Vvert}{rgb}{0.133,0.545,0.133}

      \lstset{emphstyle=\color{blue}, % les mots réservés de matlab en bleu
      basicstyle=\styleC,
      keywordstyle=\cmtt,
      commentstyle=\color{Vvert}\styleD, % commentaire en gris
      numberstyle=\tiny\color{red},
      numbers=left,
      numbersep=10pt,
      lineskip=0.7pt,
      showstringspaces=false}
      %  % inclure le fichier source
      \newcommand{\FSource}[1]{%
      \lstinputlisting[texcl=true]{#1}
      }

\setlength{\textwidth}{16cm} % Largeur du texte
\setlength{\textheight}{24cm} % Hauteur du texte
\setlength{\evensidemargin}{-0.2cm} % Taille des marges pour les pages paires
\setlength{\oddsidemargin}{-0.2cm} % Taille des marges pour les pages impaires
\setlength{\topmargin}{-1.5cm} % Taille de l'ent\^ete

\title{PR307 Smart-EcoBox : \\ Réunion 30/10/2014}

\usepackage{natbib}
\usepackage{graphicx}

\begin{document}

\maketitle

\section {Terminologies pour les tables de la base de données}
\begin{itemize}
\item \textbf{Table des valeurs}: Table où sont stockées les valeurs des capteurs et évènements connus par la table d'identification.\\
Si la trame provient d'un capteur, le format sera : (id\_capteur, timestamp, valeur). Ex : ("A45",14742585,20)\\
Si la trame est un évènement issu du serveur interprète, le format sera : (id\_evt, timestamp, valeur). Ex : ("B487E", 14742585, "ON").
\item \textbf{Table d'identification}: Table où les identifiants non existants dans la table des valeurs sont stockés.\\ 
Si la trame provient d'un capteur, le format sera : (id\_capteur, localisation, type de grandeur physique). ("A45","cuisine","Watt")\\
Si la trame est un évènement issu du serveur interprète, le format sera : (id\_evt, type\_d'appareil, marqueur\_evt).Ex :("B487E","lampe","evt").
\item \textbf{Table d'incertitude} : (Existe uniquement sur le logiciel "interaction avec le client"). Table contenant les évènements que le serveur interprète n'a pas pu identifier. Elle contient pour un évènement, n probabilités de détection de n types d'appareils. Le client est censé lire ses notifications de "doute" pour qu'on puisse régulièrement vider cette table et mettre à jour la table des valeurs.
\item \textbf{Table POI}: Table des points d'intérêts d'une courbe de charge d'un type d'appareil, utilisée par le serveur interprète. \\
Remarques : les deux dernières tables sont à retraiter.
\end{itemize}

\section{Déroulement d'un exemple pour un capteur de température}
\subsection{Phase 1}
\begin{enumerate}
\item Un capteur de température dans la cuisine est connecté à la box contenant le système.
\item Les données sont traitées par le récepteur de trames.
\item Une notification est envoyée au client web via le serveur pour le prévenir de la connexion du capteur.
\item Etant donné que le capteur est nouveau pour le système, il est stocké dans la \textbf{table d'identification}. Ex : (4882,"cuisine","Temp")
\item La capteur est maintenant enregistré dans la table d'identification. Les valeurs de température vont pouvoir être stockées dans la \textbf{table des valeurs}. Ex : ("4882",84848181,20)
\item Ces données vont être transférées au client via le serveur web qui permettront d'afficher les courbes de température.
\end{enumerate}
\subsection{Phase 2}
\begin{enumerate}
\item Un radiateur augmente la température de la cuisine, le capteur de température envoie donc ses valeurs au récepteur de trames.
\item Le récepteur des trames transmet les données du capteur au serveur interprète.
\item Le serveur interprète traite et reconnait ces valeurs et interprète un nouvel évènement lié à au radiateur qui sera stocké dans la \textbf{table d'identification}.Ex : ("Z8484","radiateur","evt")
\item Les valeurs de l'évènements sont donc stockées dans la \textbf{table des valeurs}. Ex: ("Z8484",4846584554,"ON")
\end{enumerate}

\end{document}
