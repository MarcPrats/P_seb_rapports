\documentclass[10pt,a4paper]{article}
\usepackage[utf8]{inputenc}
%\usepackage[T1]{fontenc}
\usepackage[french]{babel}
\usepackage{enumitem}
%\frenchbsetup{StandardLists=true}
\usepackage{caption}
\usepackage{graphicx}
\usepackage{amsmath}
\usepackage{subfigure}
\usepackage{appendix}
\usepackage{listings}
\usepackage{multicol}
\usepackage{moreverb}
\usepackage{epsfig}
\usepackage{textcomp}
\usepackage[table,xcdraw]{xcolor}
\usepackage{booktabs}

    \usepackage{xcolor}
      %%%%%%%%%%%%%%%%  inclure la source %%%%%%%%%%%%%%%%%%%%
  \newcommand*\styleC{\fontsize{9}{10pt}\usefont{T1}{cmtt}{m}{n}\selectfont }
  \newcommand*\styleD{\fontsize{9}{10pt}\usefont{T1}{cmtt}{m}{n}\selectfont }

      \makeatletter
      % on fixe le langage utilisé
      \lstset{language=matlab, breaklines=true}
      \edef\Motscle{emph={\lst@keywords}}
      \expandafter\lstset\expandafter{%
       \Motscle}
      \makeatother


      \definecolor{Vvert}{rgb}{0.133,0.545,0.133}

      \lstset{emphstyle=\color{blue}, % les mots réservés de matlab en bleu
      basicstyle=\styleC,
      keywordstyle=\cmtt,
      commentstyle=\color{Vvert}\styleD, % commentaire en gris
      numberstyle=\tiny\color{red},
      numbers=left,
      numbersep=10pt,
      lineskip=0.7pt,
      showstringspaces=false}
      %  % inclure le fichier source
      \newcommand{\FSource}[1]{%
      \lstinputlisting[texcl=true]{#1}
      }
% add subtitle command
\usepackage{titling}
\newcommand{\subtitle}[1]{%
  \posttitle{%
    \par\end{center}
    \begin{center}\large#1\end{center}
    \vskip0.5em}%
}

\setlength{\textwidth}{16cm} % Largeur du texte
\setlength{\textheight}{24cm} % Hauteur du texte
\setlength{\evensidemargin}{-0.2cm} % Taille des marges pour les pages paires
\setlength{\oddsidemargin}{-0.2cm} % Taille des marges pour les pages impaires
\setlength{\topmargin}{-1.5cm} % Taille de l'ent\^ete

\title{PR307 Smart-EcoBox : \\ GLRT}
\subtitle{ScrumBoard-like Client}
\author{}

\begin{document}
\maketitle
\begin{abstract}
Nous allons présenter ici les tâches à effectuer pour le développement du client. 
\end{abstract}

%%%%%%%%%%%%%%%%%%%%%%%%%%%%%%%%%%%%%%%%%%%%%%%%%%%%%%%%%%%%%%%%%%%%%%%%%%%
% \section{Organisation du groupe}

% Découpage et Organisation des fichiers

%%%%%%%%%%%%%%%%%%%%%%%%%%%%%%%%%%%%%%%%%%%%%%%%%%%%%%%%%%%%%%%%%%%%%%%%%%%
\section{Courbes}

Faire les fenêtres détaillées des courbes des capteurs qui contiendront les courbes en détails :

\begin{itemize}[label=$\bullet$]
\setlength\itemsep{1em}
    \item Affichage des courbes (Toutes les courbes, Température, Consommation électrique, débit d'eau)
    
    \item Laisser à l'utilisateur le choix de la courbe à afficher si plusieurs capteurs de même type (sur l'affichage minimaliste aka \textbf{Menu} et sur le \textbf{modal}).
    
    Exemple : si plusieurs capteurs de température, l'utilisateur doit pouvoir 
    \begin{itemize}[label=$\circ$]
    \item sur le \textbf{menu} : 1 seule allure de courbe est disponible, l'utilisateur peut choisir celle à afficher.
    \item sur le \textbf{modal} : l'utilisateur a le choix entre \textit{afficher toutes les courbes} et \textit{afficher une courbe précise}.
    \end{itemize}
\end{itemize}

%%%%%%%%%%%%%%%%%%%%%%%%%%%%%%%%%%%%%%%%%%%%%%%%%%%%%%%%%%%%%%%%%%%%%%%%%%%
\section{Notifications}

\begin{itemize}[label=$\bullet$]
\setlength\itemsep{1em}
    \item Nombre de notif limité sur le \textbf{Menu}
    
    \item Afficher toute les notifs si on clique sur \textbf{Notifications}
    
    \item Faire liste des notifications avec l'aide des ISNCs.
        \begin{itemize}[label=$\circ$]
            \item \textsc{appareil} allumée/éteint
        	\item proba1 appareil1 - proba2 appareil2 - proba3 appareil3 allumée/éteint
        	\item Personnalisation des notifications 
        	
        	exemple : Un capteur de température est situé dans la chambre de bébé, l'utilisateur souhaite être notifié si la température est en dessous de 18°C ou au dessus de 20°C.
        \end{itemize}
        
    \item L'utilisateur doit pouvoir corriger les notifications 
\end{itemize}
%%%%%%%%%%%%%%%%%%%%%%%%%%%%%%%%%%%%%%%%%%%%%%%%%%%%%%%%%%%%%%%%%%%%%%%%%%%
\section{Compte/Configuration}

\begin{itemize}[label=$\bullet$]
\setlength\itemsep{1em}
\item Créer un espace \textbf{paramètres du compte} ( édition du profil, changement du mot de passe )

\item Espace utilisateur lui permettant d'ajouter des capteurs manuellement
    \begin{itemize}[label=$\circ$]
        \item localisation
        \item type de données (T,P,D)
        \item personnalisation des notifications vis à vis des valeurs retournées par le capteur
    \end{itemize}
\end{itemize}
%%%%%%%%%%%%%%%%%%%%%%%%%%%%%%%%%%%%%%%%%%%%%%%%%%%%%%%%%%%%%%%%%%%%%%%%%%%
\section{Autres}

\begin{itemize}[label=$\bullet$]
\setlength\itemsep{1em}
\item Adapter à toutes les tailles d'écran

\item Rajouter des icônes de font awesome à côté de 
    \begin{itemize}[label=$\circ$]
        \item chaque type d'appareil
        \item chaque titre de courbes
        \item notifications
    \end{itemize}

\item Changer couleur du type d'appareil (notif + liste type d'appareils)
\end{itemize}
%%%%%%%%%%%%%%%%%%%%%%%%%%%%%%%%%%%%%%%%%%%%%%%%%%%%%%%%%%%%%%%%%%%%%%%%%%%

% \begin{thebibliography}{10}
%     \bibitem{sqlite}
%     Site officiel de sqlite : \emph{http://www.sqlite.org/}
%     \bibitem{nginx}
%     Site officiel de nginx : \emph{http://www.nginx.org/}
%     \bibitem{wiki}
%     Comparaison lighttpd VS nginx :  \emph{http://www.wikivs.com/wiki/lighttpd\_vs\_nginx}
% \end{thebibliography}

\end{document}