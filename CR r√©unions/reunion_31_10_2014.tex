\documentclass[10pt,a4paper]{article}
\usepackage[utf8]{inputenc}
%\usepackage[T1]{fontenc}
\usepackage[french]{babel}
\usepackage{enumitem}
%\frenchbsetup{StandardLists=true}
\usepackage{caption}
\usepackage{graphicx}
\usepackage{amsmath}
\usepackage{subfigure}
\usepackage{appendix}
\usepackage{listings}
\usepackage{multicol}
\usepackage{moreverb}
\usepackage{epsfig}
\usepackage{textcomp}
\usepackage[table,xcdraw]{xcolor}
\usepackage{booktabs}

    \usepackage{xcolor}
      %%%%%%%%%%%%%%%%  inclure la source %%%%%%%%%%%%%%%%%%%%
  \newcommand*\styleC{\fontsize{9}{10pt}\usefont{T1}{cmtt}{m}{n}\selectfont }
  \newcommand*\styleD{\fontsize{9}{10pt}\usefont{T1}{cmtt}{m}{n}\selectfont }

      \makeatletter
      % on fixe le langage utilisé
      \lstset{language=matlab, breaklines=true}
      \edef\Motscle{emph={\lst@keywords}}
      \expandafter\lstset\expandafter{%
       \Motscle}
      \makeatother


      \definecolor{Vvert}{rgb}{0.133,0.545,0.133}

      \lstset{emphstyle=\color{blue}, % les mots réservés de matlab en bleu
      basicstyle=\styleC,
      keywordstyle=\cmtt,
      commentstyle=\color{Vvert}\styleD, % commentaire en gris
      numberstyle=\tiny\color{red},
      numbers=left,
      numbersep=10pt,
      lineskip=0.7pt,
      showstringspaces=false}
      %  % inclure le fichier source
      \newcommand{\FSource}[1]{%
      \lstinputlisting[texcl=true]{#1}
      }

\setlength{\textwidth}{16cm} % Largeur du texte
\setlength{\textheight}{24cm} % Hauteur du texte
\setlength{\evensidemargin}{-0.2cm} % Taille des marges pour les pages paires
\setlength{\oddsidemargin}{-0.2cm} % Taille des marges pour les pages impaires
\setlength{\topmargin}{-1.5cm} % Taille de l'ent\^ete

\title{PR307 Smart-EcoBox : \\ Réunion}

\usepackage{natbib}
%\usepackage{graphicx}

\begin{document}

\maketitle

\section {Structure du projet}
\textbf{BDD :} Base de données

\begin{figure}[h]
\centering
\includegraphics[scale=0.4]{structure_projet}
\caption{schéma de l'architecture du projet}
\label{fig:my_label}
\end{figure}

\section{Organisation Table des données}

On cherche à séparer les données des capteurs des données liées aux types d'appareil utilisés.
\begin{enumerate}
\item \textbf{Identification des capteurs :} Cette table regroupe tous les capteurs connectés au système. Elle enregistre chaque nouveau capteur sous la forme (string Id, string Localisation, char Type\_de\_données ) -> ("ae485","cuisine","Temperature").
\item \textbf{Valeurs des capteurs :} Cette table regroupe les différentes valeurs enregistrées au cours 
du temps par les capteurs présents dans la table d'identification : (string Id , integer Timestamp, float Value ) -> ("ae485",1818461,22.3)
\item \textbf{États des types d'appareils : } Cette table regroupe les différents états des types d'appareils au cours du temps : (integer Id\_type\_appareil, integer Timestamp, boolean État) -> (42,1845471,1)
\item \textbf{Identification/interprétation des types d'appareils :} 
(integer Id\_type\_appareil, string Nom\_type\_appareil, integer POI1, integer POI2, integer POI3, integer POI4, integer POI5,) \textsc{¿ Sur/PasSur et AliasPourSur=0?}
\end{enumerate}

\begin{table}[h!]
    \centering
    \begin{tabular}{|l|l|l|}
    \hline
    \rowcolor[HTML]{EFEFEF} 
    string Id & string Localisation & char Type\_de\_données \\ \hline
    j46nc4    & cuisine             & T                      \\ \hline
    465hjs    & garage              & W                      \\ \hline
    hfh765    & garage              & D                      \\ \hline
              &                     &                       
    \end{tabular}
    \caption{Identification des capteurs}
\end{table}

\begin{table}[h!]
\centering
    \begin{tabular}{|l|l|l|}
    \hline
    \rowcolor[HTML]{EFEFEF} 
    string Id & integer Timestamp & integer Value \\ \hline
    j46nc4    & 1456786           & 20            \\ \hline
    465hjs    & 1456787           & 220           \\ \hline
    hfh765    & 1456788           & 10            \\ \hline
    j46nc4    & 1456790           & 18            \\ \hline
    j46nc4    & 1456792           & 24            \\ \hline
    hfh765    & 1456795           & 11            \\ \hline
    hfh765    & 1456798           & 12            \\ \hline
              &                   &              
    \end{tabular}
    \caption{Valeurs des capteurs}
\end{table}

\begin{table}[h!]
\resizebox{\textwidth}{!}{%
\begin{tabular}{|l|l|l|l|l|l|l|}
\hline
\rowcolor[HTML]{EFEFEF} 
\begin{tabular}[c]{@{}l@{}}integer \\ Id\_type\_appareils\end{tabular} & string Nom\_type\_appareils & integer POI1 & integer POI2 & integer POI3 & integer POI4 & integer POI5 \\ \hline
hfh584                                                                 & chauffage                   & 10           & 50           & 221          & 45           & 242          \\ \hline
hrs456                                                                 & unknown                     & 42           & 55           & 633          & 55           & 122          \\ \hline
kjn9h4                                                                 & frigo                       & 58           & 15           & 6            & 6841         & 187          \\ \hline
153gdj                                                                 & cuisinière                  & 7            & 6            & 68           & 71           & 6845         \\ \hline
846ghd                                                                 & unknown                     & 75           & 13           & 1364         & 145          & 165          \\ \hline
                                                                       &                             &              &              &              &              &             
\end{tabular}
}
\end{table}

\begin{table}[h!]
\resizebox{\textwidth}{!}{%
\begin{tabular}{|l|l|l|l|l|l|l|l|}
\hline
\rowcolor[HTML]{EFEFEF} 
boolean Checked & integer Alias & integer Numero\_type\_appareils1 & integer Proba1 & integer Numero\_type\_appareils2 & integer Proba2 & integer Numero\_type\_appareils3 & integer Proba3 \\ \hline
1               &               &                                  &                &                                  &                &                                  &                \\ \hline
0               & -1            & kjn9h4                           & 33             & hfh584                           & 27             & 153gdj                           & 25             \\ \hline
1               &               &                                  &                &                                  &                &                                  &                \\ \hline
1               &               &                                  &                &                                  &                &                                  &                \\ \hline
0               & hfh584        & hfh584                           & 33             & kjn9h4                           & 31             & 153gdj                           & 30             \\ \hline
                &               &                                  &                &                                  &                &                                  &               
\end{tabular}
}\caption{Identification/interprétation des types d'appareils}
\end{table}

\begin{table}[h!]
\centering
    \begin{tabular}{|l|l|l|}
    \hline
    \rowcolor[HTML]{EFEFEF} 
    integer Id\_type\_appareils & integer Timestamp & boolean State \\ \hline
    846ghd                               & 1456786           & 1             \\ \hline
    hfh584                               & 1456787           & 0             \\ \hline
    hfh584                               & 1456788           & 1             \\ \hline
    kjn9h4                               & 1456790           & 1             \\ \hline
    hrs456                               & 1456792           & 0             \\ \hline
    hfh584                               & 1456795           & 0             \\ \hline
    153gdj                               & 1456798           & 1             \\ \hline
                                    &                   &              
    \end{tabular}
    \caption{États des appareils}
\end{table}
\newpage
Voici les tables.
\newpage
\section{Scénario}
Au démarrage de la Box la base de donnée est vide. La Box se met en route en détectant les capteurs environnants, ce qui remplie la table d'\textit{identification des capteurs} ainsi que les premières valeurs de la table\textit{ valeurs des capteurs}. Parallèlement, la table \textit{Identification/interprétation des types d'appareils} est remplie avec les type d'appareil déjà connu par nos recherches, le champ \textbf{Checked} est à 0. \\
\begin{table}[h!]
    \centering
    \begin{tabular}{|l|l|l|}
    \hline
    \rowcolor[HTML]{EFEFEF} 
    string Id & string Localisation & char Type\_de\_données \\ \hline
              &                     &                       
    \end{tabular}
    \caption{Identification des capteurs}
\end{table}

\begin{table}[h!]
\centering
    \begin{tabular}{|l|l|l|}
    \hline
    \rowcolor[HTML]{EFEFEF} 
    string Id & integer Timestamp & integer Value \\ \hline
              &                   &              
    \end{tabular}
    \caption{Valeurs des capteurs}
\end{table}
\begin{table}[h!]
\resizebox{\textwidth}{!}{%
\begin{tabular}{|l|l|l|l|l|l|l|}
\hline
\rowcolor[HTML]{EFEFEF} 
\begin{tabular}[c]{@{}l@{}}integer \\ Id\_type\_appareils\end{tabular} & string Nom\_type\_appareils & integer POI1 & integer POI2 & integer POI3 & integer POI4 & integer POI5 \\ \hline
hfh584                                                                 & chauffage                   & 10           & 50           & 221          & 45           & 242          \\ \hline
kjn9h4                                                                 & frigo                       & 58           & 15           & 6            & 6841         & 187          \\ \hline
153gdj                                                                 & cuisinière                  & 7            & 6            & 68           & 71           & 6845         \\ \hline
                                                                 &                      &            &            &          &           &           \\ \hline
                                                                       &                             &              &              &              &              &     \\ \hline        
                                                                 &                      &            &            &           &            &           
\end{tabular}
}
\end{table}

\begin{table}[h!]
\resizebox{\textwidth}{!}{%
\begin{tabular}{|l|l|l|l|l|l|l|l|}
\hline
\rowcolor[HTML]{EFEFEF} 
boolean Checked & integer Alias & integer Numero\_type\_appareils1 & integer Proba1 & integer Numero\_type\_appareils2 & integer Proba2 & integer Numero\_type\_appareils3 & integer Proba3 \\ \hline
0               &               &                                  &                &                                  &                &                                  &                \\ \hline
0               &             &                            &              &                            &              &                            &              \\ \hline
0               &               &                                  &                &                                  &                &                                  &                \\ \hline
              &               &                                  &                &                                  &                &                                  &                \\ \hline
               &         &                            &              &                            &              &                            &              \\ \hline
                &               &                                  &                &                                  &                &                                  &               
\end{tabular}
}\caption{Identification/interprétation des types d'appareils}
\end{table}

\newpage
L'interprétation des valeurs capteurs peut alors démarrer. Les premiers événements d'appareils sont enregistrés dans la table \textit{États des appareil} et dans la table \textit{Identification/interprétation des types d'appareils} le champ \textbf{Checked} de ces appareils passent à 1. \\

\begin{table}[h!]
    \centering
    \begin{tabular}{|l|l|l|}
    \hline
    \rowcolor[HTML]{EFEFEF} 
    string Id & string Localisation & char Type\_de\_données \\ \hline
    j46nc4    & cuisine             & T                      \\ \hline
    465hjs    & garage              & W                      \\ \hline
    hfh765    & garage              & D                      \\ \hline
              &                     &                       
    \end{tabular}
    \caption{Identification des capteurs}
\end{table}

\begin{table}[h!]
\centering
    \begin{tabular}{|l|l|l|}
    \hline
    \rowcolor[HTML]{EFEFEF} 
    string Id & integer Timestamp & integer Value \\ \hline
    j46nc4    & 1456786           & 20            \\ \hline
    465hjs    & 1456787           & 220           \\ \hline
    hfh765    & 1456788           & 10            \\ \hline
    j46nc4    & 1456790           & 18            \\ \hline
    j46nc4    & 1456792           & 24            \\ \hline
    hfh765    & 1456795           & 11            \\ \hline
    hfh765    & 1456798           & 12            \\ \hline
              &                   &              
    \end{tabular}
    \caption{Valeurs des capteurs}
\end{table}
\begin{table}[h!]
\resizebox{\textwidth}{!}{%
\begin{tabular}{|l|l|l|l|l|l|l|}
\hline
\rowcolor[HTML]{EFEFEF} 
\begin{tabular}[c]{@{}l@{}}integer \\ Id\_type\_appareils\end{tabular} & string Nom\_type\_appareils & integer POI1 & integer POI2 & integer POI3 & integer POI4 & integer POI5 \\ \hline
hfh584                                                                 & chauffage                   & 10           & 50           & 221          & 45           & 242          \\ \hline
kjn9h4                                                                 & frigo                       & 58           & 15           & 6            & 6841         & 187          \\ \hline
153gdj                                                                 & cuisinière                  & 7            & 6            & 68           & 71           & 6845         \\ \hline
                                                                 &                      &            &            &          &           &           \\ \hline
                                                                       &                             &              &              &              &              &     \\ \hline        
                                                                 &                      &            &            &           &            &           
\end{tabular}
}
\end{table}

\begin{table}[h!]
\resizebox{\textwidth}{!}{%
\begin{tabular}{|l|l|l|l|l|l|l|l|}
\hline
\rowcolor[HTML]{EFEFEF} 
boolean Checked & integer Alias & integer Numero\_type\_appareils1 & integer Proba1 & integer Numero\_type\_appareils2 & integer Proba2 & integer Numero\_type\_appareils3 & integer Proba3 \\ \hline
1               &               &                                  &                &                                  &                &                                  &                \\ \hline
1               &             &                            &              &                            &              &                            &              \\ \hline
0               &               &                                  &                &                                  &                &                                  &                \\ \hline
              &               &                                  &                &                                  &                &                                  &                \\ \hline
               &         &                            &              &                            &              &                            &              \\ \hline
                &               &                                  &                &                                  &                &                                  &               
\end{tabular}
}\caption{Identification/interprétation des types d'appareils}
\end{table}
\begin{table}[h!]
\centering
    \begin{tabular}{|l|l|l|}
    \hline
    \rowcolor[HTML]{EFEFEF} 
    integer Id\_type\_appareils & integer Timestamp & boolean State \\ \hline
    hfh584                               & 1456787           & 0             \\ \hline
    hfh584                               & 1456788           & 1             \\ \hline
    kjn9h4                               & 1456790           & 1             \\ \hline
    hfh584                               & 1456795           & 0             \\ \hline
    153gdj                               & 1456798           & 1             \\ \hline	 &                   &              
    \end{tabular}
    \caption{États des appareils}
\end{table}

\newpage
L'interprétation est indécise sur le type d'un appareil. En effet elle hésite entre un chauffage $30\%$, un four $31\%$ et la voiture plug-in qui charge $33\%$. On va donc présenter à l'utilisateur ces trois probabilités sur son interface et lui permettre de résoudre le doute. Dans les tables de la base de données cela s'apparentera à ceci :\\

\begin{table}[h!]
\resizebox{\textwidth}{!}{%
\begin{tabular}{|l|l|l|l|l|l|l|}
\hline
\rowcolor[HTML]{EFEFEF} 
\begin{tabular}[c]{@{}l@{}}integer \\ Id\_type\_appareils\end{tabular} & string Nom\_type\_appareils & integer POI1 & integer POI2 & integer POI3 & integer POI4 & integer POI5 \\ \hline
hfh584                                                                 & chauffage                   & 10           & 50           & 221          & 45           & 242          \\ \hline
hrs456                                                                 & unknown                     & 42           & 55           & 633          & 55           & 122          \\ \hline
kjn9h4                                                                 & frigo                       & 58           & 15           & 6            & 6841         & 187          \\ \hline
153gdj                                                                 & cuisinière                  & 7            & 6            & 68           & 71           & 6845         \\ \hline
                                                                 &                      &            &            &          &           &           \\ \hline
                                                                       &                             &              &              &              &              &             
\end{tabular}
}
\end{table}

\begin{table}[h!]
\resizebox{\textwidth}{!}{%
\begin{tabular}{|l|l|l|l|l|l|l|l|}
\hline
\rowcolor[HTML]{EFEFEF} 
boolean Checked & integer Alias & integer Numero\_type\_appareils1 & integer Proba1 & integer Numero\_type\_appareils2 & integer Proba2 & integer Numero\_type\_appareils3 & integer Proba3 \\ \hline
1               &               &                                  &                &                                  &                &                                  &                \\ \hline
1               & -1            & kjn9h4                           & 33             & hfh584                           & 27             & 153gdj                           & 25             \\ \hline
1               &               &                                  &                &                                  &                &                                  &                \\ \hline
1               &               &                                  &                &                                  &                &                                  &                \\ \hline
               &         &                            &              &                            &              &                            &              \\ \hline
                &               &                                  &                &                                  &                &                                  &               
\end{tabular}
}\caption{Identification/interprétation des types d'appareils}
\end{table}
\newpage
L'utilisateur va résoudre le conflit en associant cet appareil inconnu à un chauffage. On va alors placer le champ alias de l'appareil inconnu de $-1$ à hfh584. Cet appareil inconnu sera désormais apparenté à un chauffage.\\

\begin{table}[h!]
\resizebox{\textwidth}{!}{%
\begin{tabular}{|l|l|l|l|l|l|l|}
\hline
\rowcolor[HTML]{EFEFEF} 
\begin{tabular}[c]{@{}l@{}}integer \\ Id\_type\_appareils\end{tabular} & string Nom\_type\_appareils & integer POI1 & integer POI2 & integer POI3 & integer POI4 & integer POI5 \\ \hline
hfh584                                                                 & chauffage                   & 10           & 50           & 221          & 45           & 242          \\ \hline
hrs456                                                                 & unknown                     & 42           & 55           & 633          & 55           & 122          \\ \hline
kjn9h4                                                                 & frigo                       & 58           & 15           & 6            & 6841         & 187          \\ \hline
153gdj                                                                 & cuisinière                  & 7            & 6            & 68           & 71           & 6845         \\ \hline
                                                                 &                      &            &            &          &           &           \\ \hline
                                                                       &                             &              &              &              &              &             
\end{tabular}
}
\end{table}

\begin{table}[h!]
\resizebox{\textwidth}{!}{%
\begin{tabular}{|l|l|l|l|l|l|l|l|}
\hline
\rowcolor[HTML]{EFEFEF} 
boolean Checked & integer Alias & integer Numero\_type\_appareils1 & integer Proba1 & integer Numero\_type\_appareils2 & integer Proba2 & integer Numero\_type\_appareils3 & integer Proba3 \\ \hline
1               &               &                                  &                &                                  &                &                                  &                \\ \hline
1               & hfh584            & kjn9h4                           & 33             & hfh584                           & 27             & 153gdj                           & 25             \\ \hline
1               &               &                                  &                &                                  &                &                                  &                \\ \hline
1               &               &                                  &                &                                  &                &                                  &                \\ \hline
               &         &                            &              &                            &              &                            &              \\ \hline
                &               &                                  &                &                                  &                &                                  &               
\end{tabular}
}\caption{Identification/interprétation des types d'appareils}
\end{table}

\newpage
L'interprétation est indécise sur le type d'un appareil. En effet elle hésite entre un chauffage $28\%$, un four $29\%$ et la voiture plug-in qui charge $30\%$. On va donc à nouveau présenter à l'utilisateur ces trois probabilités sur son interface et lui permettre de résoudre le doute. Dans les tables de la base de données cela s'apparentera à ceci :\\

\begin{table}[h!]
\resizebox{\textwidth}{!}{%
\begin{tabular}{|l|l|l|l|l|l|l|}
\hline
\rowcolor[HTML]{EFEFEF} 
\begin{tabular}[c]{@{}l@{}}integer \\ Id\_type\_appareils\end{tabular} & string Nom\_type\_appareils & integer POI1 & integer POI2 & integer POI3 & integer POI4 & integer POI5 \\ \hline
hfh584                                                                 & chauffage                   & 10           & 50           & 221          & 45           & 242          \\ \hline
hrs456                                                                 & unknown                     & 42           & 55           & 633          & 55           & 122          \\ \hline
kjn9h4                                                                 & frigo                       & 58           & 15           & 6            & 6841         & 187          \\ \hline
153gdj                                                                 & cuisinière                  & 7            & 6            & 68           & 71           & 6845         \\ \hline
846ghd                                                                 & unknown                     & 75           & 13           & 1364         & 145          & 165          \\ \hline
                                                                       &                             &              &              &              &              &             
\end{tabular}
}
\end{table}

\begin{table}[h!]
\resizebox{\textwidth}{!}{%
\begin{tabular}{|l|l|l|l|l|l|l|l|}
\hline
\rowcolor[HTML]{EFEFEF} 
boolean Checked & integer Alias & integer Numero\_type\_appareils1 & integer Proba1 & integer Numero\_type\_appareils2 & integer Proba2 & integer Numero\_type\_appareils3 & integer Proba3 \\ \hline
1               &               &                                  &                &                                  &                &                                  &                \\ \hline
1               & hfh584            & kjn9h4                           & 33             & hfh584                           & 27             & 153gdj                           & 25             \\ \hline
1               &               &                                  &                &                                  &                &                                  &                \\ \hline
1               &               &                                  &                &                                  &                &                                  &                \\ \hline
1               & -1        & hfh584                           & 33             & kjn9h4                           & 31             & 153gdj                           & 30             \\ \hline
                &               &                                  &                &                                  &                &                                  &               
\end{tabular}
}\caption{Identification/interprétation des types d'appareils}
\end{table}

\newpage
L'utilisateur va résoudre le conflit en associant cet appareil inconnu à un nouvel appareil non référencé : TV. On va alors placer le champ alias de l'appareil inconnu de $-1$ à $0$, changer le nom à TV.  

\begin{table}[h!]
\resizebox{\textwidth}{!}{%
\begin{tabular}{|l|l|l|l|l|l|l|}
\hline
\rowcolor[HTML]{EFEFEF} 
\begin{tabular}[c]{@{}l@{}}integer \\ Id\_type\_appareils\end{tabular} & string Nom\_type\_appareils & integer POI1 & integer POI2 & integer POI3 & integer POI4 & integer POI5 \\ \hline
hfh584                                                                 & chauffage                   & 10           & 50           & 221          & 45           & 242          \\ \hline
hrs456                                                                 & unknown                     & 42           & 55           & 633          & 55           & 122          \\ \hline
kjn9h4                                                                 & frigo                       & 58           & 15           & 6            & 6841         & 187          \\ \hline
153gdj                                                                 & cuisinière                  & 7            & 6            & 68           & 71           & 6845         \\ \hline
846ghd                                                                 & TV                     & 75           & 13           & 1364         & 145          & 165          \\ \hline
                                                                       &                             &              &              &              &              &             
\end{tabular}
}
\end{table}

\begin{table}[h!]
\resizebox{\textwidth}{!}{%
\begin{tabular}{|l|l|l|l|l|l|l|l|}
\hline
\rowcolor[HTML]{EFEFEF} 
boolean Checked & integer Alias & integer Numero\_type\_appareils1 & integer Proba1 & integer Numero\_type\_appareils2 & integer Proba2 & integer Numero\_type\_appareils3 & integer Proba3 \\ \hline
1               &               &                                  &                &                                  &                &                                  &                \\ \hline
1               & hfh584            & kjn9h4                           & 33             & hfh584                           & 27             & 153gdj                           & 25             \\ \hline
1               &               &                                  &                &                                  &                &                                  &                \\ \hline
1               &               &                                  &                &                                  &                &                                  &                \\ \hline
1               & 0        & hfh584                           & 33             & kjn9h4                           & 31             & 153gdj                           & 30             \\ \hline
                &               &                                  &                &                                  &                &                                  &               
\end{tabular}
}\caption{Identification/interprétation des types d'appareils}
\end{table}


\section{Choix des technologies}
Nos contraintes pour les données sont les suivantes :
\begin{itemize}
\item La transmission des données est synchrone (les données vont être transmises successivement) ;
\item Les données sont triées, opérations de jointures sur les tables -> Besoin de base de données relationnelles ;
\item Les tecnhologies doivent consommer le moins de ressources possibles pour fonctionner sur un dispositif tel qu'un Rasperry Pi ;
\end{itemize}   
\textbf{Propositions de technologie}:
\begin{itemize}
\item \textbf{Base de données SQLite : }Cette technologie légère et relationnelle peut être contenue dans un seul fichier. 
\item \textbf{Serveur Nginx ou lighttpd : }Ces serveurs demandent peu de resources et s'adaptent aux bases relationnelles.
\item \textbf{Langages PHP, J2EE, Ruby : }Ces langages natifs sont les plus adaptés aux serveurs énoncés précédemment.
\end{itemize}
\section{Organisation GLRT}
\begin{itemize}
\item Lien entre le récepteur/interprète et la base de données
\item Lien entre la base de données et le client web via le serveur
\item Client Web : Vues + Calculs
\end{itemize}



\end{document}