\documentclass[10pt,a4paper]{article}
\usepackage[utf8]{inputenc}
%\usepackage[T1]{fontenc}
\usepackage[french]{babel}
\usepackage{enumitem}
%\frenchbsetup{StandardLists=true}
\usepackage{caption}
\usepackage{graphicx}
\usepackage{amsmath}
\usepackage{subfigure}
\usepackage{appendix}
\usepackage{listings}
\usepackage{multicol}
\usepackage{moreverb}
\usepackage{epsfig}
\usepackage{textcomp}
\usepackage[table,xcdraw]{xcolor}
\usepackage{booktabs}

    \usepackage{xcolor}
      %%%%%%%%%%%%%%%%  inclure la source %%%%%%%%%%%%%%%%%%%%
  \newcommand*\styleC{\fontsize{9}{10pt}\usefont{T1}{cmtt}{m}{n}\selectfont }
  \newcommand*\styleD{\fontsize{9}{10pt}\usefont{T1}{cmtt}{m}{n}\selectfont }

      \makeatletter
      % on fixe le langage utilisé
      \lstset{language=matlab, breaklines=true}
      \edef\Motscle{emph={\lst@keywords}}
      \expandafter\lstset\expandafter{%
       \Motscle}
      \makeatother


      \definecolor{Vvert}{rgb}{0.133,0.545,0.133}

      \lstset{emphstyle=\color{blue}, % les mots réservés de matlab en bleu
      basicstyle=\styleC,
      keywordstyle=\cmtt,
      commentstyle=\color{Vvert}\styleD, % commentaire en gris
      numberstyle=\tiny\color{red},
      numbers=left,
      numbersep=10pt,
      lineskip=0.7pt,
      showstringspaces=false}
      %  % inclure le fichier source
      \newcommand{\FSource}[1]{%
      \lstinputlisting[texcl=true]{#1}
      }

\setlength{\textwidth}{16cm} % Largeur du texte
\setlength{\textheight}{24cm} % Hauteur du texte
\setlength{\evensidemargin}{-0.2cm} % Taille des marges pour les pages paires
\setlength{\oddsidemargin}{-0.2cm} % Taille des marges pour les pages impaires
\setlength{\topmargin}{-1.5cm} % Taille de l'ent\^ete

\title{PR307 Smart-EcoBox : \\ Réunion}

\usepackage{natbib}
%\usepackage{graphicx}


\begin{document}
\maketitle

\begin{figure}[ht]
\begin{center}
\noindent \includegraphics[scale = 0.25]{../../Logo_ENSEIRB.pdf}
\end{center}
\end{figure}
\begin{figure}[ht]
\begin{center}
\noindent \includegraphics[scale = 0.25]{../../smartee_logo.pdf}
\end{center}
\end{figure}


\newpage

\tableofcontents
\newpage
\section {Structure du projet}

\begin{figure}[h]
\centering
\includegraphics[scale=0.4]{structure_projet}
\caption{schéma de l'architecture du projet}
\label{fig:my_label}
\end{figure}

\textbf{BDD :} Base de données
\section{Organisation Table des données}

On cherche à séparer les données des capteurs des données liées aux types d'appareil utilisés.
\begin{enumerate}
\item \textbf{Identification des capteurs :} Cette table regroupe tous les capteurs connectés au système. Elle enregistre chaque nouveau capteur.
\begin{table}[h!]
    \centering
    \begin{tabular}{|l|l|l|}
    \hline
    \rowcolor[HTML]{EFEFEF} 
    string Id & string Localisation & char Type\_de\_données \\ \hline
    j46nc4    & cuisine             & T                      \\ \hline
    465hjs    & garage              & W                      \\ \hline
    hfh765    & garage              & D                      \\ \hline
              &                     &                       
    \end{tabular}
    \caption{Identification des capteurs}
\end{table}
\\
\begin{itemize}
\item \textbf{Id} : unique lié à chaque capteur.
\item \textbf{Localisation} : Permet à l'utilisateur de renseigner une "Localisation" du capteur.
\item \textbf{Type de données} : Permet de définir si le capteur renvoie une température (T), une puissance (W) ou un débit d'eau (D).
\end{itemize}
\newpage
\item \textbf{Valeurs des capteurs :} Cette table regroupe les différentes valeurs enregistrées au cours du temps par les capteurs présents dans la table d'identification.
\begin{table}[h!]
\centering
    \begin{tabular}{|l|l|l|}
    \hline
    \rowcolor[HTML]{EFEFEF} 
    string Id & integer Timestamp & integer Value \\ \hline
    j46nc4    & 1456786           & 20            \\ \hline
    465hjs    & 1456787           & 220           \\ \hline
    hfh765    & 1456788           & 10            \\ \hline
    j46nc4    & 1456790           & 18            \\ \hline
    j46nc4    & 1456792           & 24            \\ \hline
    hfh765    & 1456795           & 11            \\ \hline
    hfh765    & 1456798           & 12            \\ \hline
              &                   &              
    \end{tabular}
    \caption{Valeurs des capteurs}
\end{table}
\begin{itemize}
\item \textbf{Id} : unique lié à chaque capteur.
\item \textbf{Timestamp} : Horodatage de chaque valeur stockée
\item \textbf{Valeur} : Stocke les valeurs, l'unité est donnée par correspondance avec la table \textbf{identification des capteurs}
\end{itemize}
\item \textbf{États des types d'appareils : } Cette table regroupe les différents états des types d'appareils au cours du temps.\\

\begin{table}[h!]
\centering
    \begin{tabular}{|l|l|l|}
    \hline
    \rowcolor[HTML]{EFEFEF} 
    integer Id\_type\_appareils & integer Timestamp & boolean State \\ \hline
    1       & 1456787           & 0             \\ \hline
    1       & 1456788           & 1             \\ \hline
    2       & 1456790           & 1             \\ \hline
    1       & 1456795           & 0             \\ \hline
    3       & 1456798           & 1             \\ \hline
    10000       & 1456799          & 0             \\ \hline
    10001       & 1456800           & 1             \\ \hline
                 &                   &              
    \end{tabular}
    \caption{États des types d'appareils}
\end{table} 
\begin{itemize}
\item \textbf{Id\_type\_appareils} : unique, provient soit d'une base de données déjà créée grâce aux recherches, soit créée par l'utilisateur lors d'un type d'appareil inconnu.
\item \textbf{Timestamp} : Horodatage de chaque valeur stockée.
\item \textbf{State} : Allumé : 1, éteint : 0.
\end{itemize}
\item \textbf{Identification/interprétation des types d'appareils :} Cette table regroupe les différents types d'appareils identifiés, les points d'intérêt associés et les notions nécessaires à la gestion de l'incertitude des interprétations.
\begin{table}[h]
\resizebox{\textwidth}{!}{%
\begin{tabular}{|c|c|c|c|c|c|c|c|c|c|c|c|c|c|c|}
\hline
\rowcolor[HTML]{EFEFEF} 
\begin{tabular}[c]{@{}c@{}}integer \\ Id type \\ appareils\end{tabular} & \begin{tabular}[c]{@{}c@{}}string \\ Nom type\\  appareils\end{tabular} & \begin{tabular}[c]{@{}c@{}}integer\\  POI1\end{tabular} & \begin{tabular}[c]{@{}c@{}}integer\\  POI2\end{tabular} & \begin{tabular}[c]{@{}c@{}}integer\\  POI3\end{tabular} & \begin{tabular}[c]{@{}c@{}}integer\\  POI4\end{tabular} & \begin{tabular}[c]{@{}c@{}}integer\\  POI5\end{tabular} & \begin{tabular}[c]{@{}c@{}}boolean\\  checked\end{tabular} & \begin{tabular}[c]{@{}c@{}}integer\\  Alias\end{tabular} & \begin{tabular}[c]{@{}c@{}}integer \\ Numero\\  type \\ appareil 1\end{tabular} & \begin{tabular}[c]{@{}c@{}}integer\\  Proba 1\end{tabular} & \begin{tabular}[c]{@{}c@{}}integer \\ Numero\\  type \\ appareil 2\end{tabular} & \begin{tabular}[c]{@{}c@{}}integer \\ Proba 2\end{tabular} & \begin{tabular}[c]{@{}c@{}}integer \\ Numero\\  type \\ appareil 3\end{tabular} & \begin{tabular}[c]{@{}c@{}}integer \\ Proba 3\end{tabular} \\ \hline
1 & chauffage & 10 & 15 & 20 & 49 & 200 & 1 & 0 &  &  &  &  &  &  \\ \hline
2 & frigo & 139 & 3829 & 1329 & 0284 & 1209 & 1 & 0 &  &  &  &  &  &  \\ \hline
3 & cuisinière & 17 & 8364 & 1893 & 1360 & 1361 & 1 & 0  &  &  &  &  &  &  \\ \hline
10000 & unknown & 1073 & 13 & 324 & 2486 & 1846 & 0 & 1 & 2 & 30 & 1 & 27 & 3 & 25 \\ \hline
10001 & TV & 328 & 2 & 2487 & 23 & 427 & 0 & 0 & 3 & 29 & 1 & 27 & 2 & 22 \\ \hline
 &  &  &  &  &  &  &  &  &  &  &  &  &  & 
\end{tabular}
}
\caption{Identification/interprétation des types d'appareils}
\end{table} 
\begin{itemize}
\item \textbf{Id} : unique lié à chaque type d'appareil.
\item \textbf{Nom type d'appareil} : nom défini par nos recherches ou par l'utilisateur dans le cas de type d'appareil inconnu.
\item \textbf{$POI_n$} : points d'intérêts nécessaires à l'identification des appareils à partir des données de puissance.
\item \textbf{checked} : à 0 quand l'appareil de ce type n'a jamais été détecté. Passe à 1 quand un appareil de ce type est détecté. Permet la séparation des données produites par la recherche et celles qui sont utilisées. Le scénario apporte plus d'informations à ce sujet.
\item \textbf{Alias} permet de renvoyer à un autre type d'appareil dans le cas de plusieurs POI pour le même type d'appareil.
\item \textbf{Champs suivants} ils sont détaillés dans le scénario.
\end{itemize}
\end{enumerate}
\newpage
\section{Scénario}
\subsection{Initialisation}
Au démarrage de la Box la base de données est vide. La Box se met en route en détectant les capteurs environnants, ce qui remplit la table d'\textit{identification des capteurs} ainsi que les premières valeurs de la table\textit{ valeurs des capteurs}. Parallèlement, la table \textit{Identification/interprétation des types d'appareils} est remplie avec les types d'appareils déjà connus par nos recherches, le champ \textbf{Checked} est à 0. \\
\begin{table}[h!]
    \centering
    \begin{tabular}{|l|l|l|}
    \hline
    \rowcolor[HTML]{EFEFEF} 
    string Id & string Localisation & char Type\_de\_données \\ \hline
    j46nc4    & cuisine             & T                      \\ \hline
    465hjs    & garage              & W                      \\ \hline
    hfh765    & garage              & D                      \\ \hline
              &                     &                       
    \end{tabular}
    \caption{Identification des capteurs}
\end{table}

\begin{table}[h!]
\centering
    \begin{tabular}{|l|l|l|}
    \hline
    \rowcolor[HTML]{EFEFEF} 
    string Id & integer Timestamp & integer Value \\ \hline
    j46nc4    & 1456786           & 20            \\ \hline
    465hjs    & 1456787           & 220           \\ \hline
    hfh765    & 1456788           & 10            \\ \hline
    j46nc4    & 1456790           & 18            \\ \hline
    j46nc4    & 1456792           & 24            \\ \hline
    hfh765    & 1456795           & 11            \\ \hline
    hfh765    & 1456798           & 12            \\ \hline
              &                   &              
    \end{tabular}
    \caption{Valeurs des capteurs}
\end{table}

\begin{table}[h]
\resizebox{\textwidth}{!}{%
\begin{tabular}{|c|c|c|c|c|c|c|c|c|c|c|c|c|c|c|}
\hline
\rowcolor[HTML]{EFEFEF} 
\begin{tabular}[c]{@{}c@{}}integer \\ Id type \\ appareils\end{tabular} & \begin{tabular}[c]{@{}c@{}}string \\ Nom type\\  appareils\end{tabular} & \begin{tabular}[c]{@{}c@{}}integer\\  POI1\end{tabular} & \begin{tabular}[c]{@{}c@{}}integer\\  POI2\end{tabular} & \begin{tabular}[c]{@{}c@{}}integer\\  POI3\end{tabular} & \begin{tabular}[c]{@{}c@{}}integer\\  POI4\end{tabular} & \begin{tabular}[c]{@{}c@{}}integer\\  POI5\end{tabular} & \begin{tabular}[c]{@{}c@{}}boolean\\  checked\end{tabular} & \begin{tabular}[c]{@{}c@{}}integer\\  Alias\end{tabular} & \begin{tabular}[c]{@{}c@{}}integer \\ Numero\\  type \\ appareil 1\end{tabular} & \begin{tabular}[c]{@{}c@{}}integer\\  Proba 1\end{tabular} & \begin{tabular}[c]{@{}c@{}}integer \\ Numero\\  type \\ appareil 2\end{tabular} & \begin{tabular}[c]{@{}c@{}}integer \\ Proba 2\end{tabular} & \begin{tabular}[c]{@{}c@{}}integer \\ Numero\\  type \\ appareil 3\end{tabular} & \begin{tabular}[c]{@{}c@{}}integer \\ Proba 3\end{tabular} \\ \hline
1 & chauffage & 10 & 15 & 20 & 49 & 200 & 0 & 0 &  &  &  &  &  &  \\ \hline
2 & frigo & 139 & 3829 & 1329 & 0284 & 1209 & 0 & 0 &  &  &  &  &  &  \\ \hline
3 & cuisinière & 17 & 8364 & 1893 & 1360 & 1361 & 0 & 0  &  &  &  &  &  &  \\ \hline
 &  &  &  &  &  &  &  &  &  &  &  &  &  & 
\end{tabular}
}
\caption{Identification/interprétation des types d'appareils}
\end{table}
\begin{table}[h!]
\centering
    \begin{tabular}{|l|l|l|}
    \hline
    \rowcolor[HTML]{EFEFEF} 
    integer Id\_type\_appareils & integer Timestamp & boolean State \\ \hline                 &                   &              
    \end{tabular}
    \caption{États des appareils}
\end{table}
\newpage
\subsection{Premiers changements d'état d'appareils}
L'interprétation des valeurs capteurs peut alors démarrer. Les premiers événements d'appareils sont enregistrés dans la table \textit{États des appareil} et dans la table \textit{Identification/interprétation des types d'appareils} le champ \textbf{Checked} de ces appareils passent à 1. \\


\begin{table}[h]
\resizebox{\textwidth}{!}{%
\begin{tabular}{|c|c|c|c|c|c|c|c|c|c|c|c|c|c|c|}
\hline
\rowcolor[HTML]{EFEFEF} 
\begin{tabular}[c]{@{}c@{}}integer \\ Id type \\ appareils\end{tabular} & \begin{tabular}[c]{@{}c@{}}string \\ Nom type\\  appareils\end{tabular} & \begin{tabular}[c]{@{}c@{}}integer\\  POI1\end{tabular} & \begin{tabular}[c]{@{}c@{}}integer\\  POI2\end{tabular} & \begin{tabular}[c]{@{}c@{}}integer\\  POI3\end{tabular} & \begin{tabular}[c]{@{}c@{}}integer\\  POI4\end{tabular} & \begin{tabular}[c]{@{}c@{}}integer\\  POI5\end{tabular} & \begin{tabular}[c]{@{}c@{}}boolean\\  checked\end{tabular} & \begin{tabular}[c]{@{}c@{}}integer\\  Alias\end{tabular} & \begin{tabular}[c]{@{}c@{}}integer \\ Numero\\  type \\ appareil 1\end{tabular} & \begin{tabular}[c]{@{}c@{}}integer\\  Proba 1\end{tabular} & \begin{tabular}[c]{@{}c@{}}integer \\ Numero\\  type \\ appareil 2\end{tabular} & \begin{tabular}[c]{@{}c@{}}integer \\ Proba 2\end{tabular} & \begin{tabular}[c]{@{}c@{}}integer \\ Numero\\  type \\ appareil 3\end{tabular} & \begin{tabular}[c]{@{}c@{}}integer \\ Proba 3\end{tabular} \\ \hline
1 & chauffage & 10 & 15 & 20 & 49 & 200 & 1 & 0 &  &  &  &  &  &  \\ \hline
2 & frigo & 139 & 3829 & 1329 & 0284 & 1209 & 1 & 0 &  &  &  &  &  &  \\ \hline
3 & cuisinière & 17 & 8364 & 1893 & 1360 & 1361 & 1 & 0  &  &  &  &  &  &  \\ \hline
 &  &  &  &  &  &  &  &  &  &  &  &  &  & 
\end{tabular}
}
\caption{Identification/interprétation des types d'appareils}
\end{table}
\begin{table}[h!]
\centering
    \begin{tabular}{|l|l|l|}
    \hline
    \rowcolor[HTML]{EFEFEF} 
    integer Id\_type\_appareils & integer Timestamp & boolean State \\ \hline
    1       & 1456787           & 0             \\ \hline
    1       & 1456788           & 1             \\ \hline
    2       & 1456790           & 1             \\ \hline
    1       & 1456795           & 0             \\ \hline
    3       & 1456798           & 1             \\ \hline
                 &                   &              
    \end{tabular}
    \caption{États des appareils}
\end{table} 

\newpage
\subsection{Appareil inconnu détecté}
\subsubsection{Cas 1}
L'interprétation est indécise sur le type d'un appareil. En effet, elle hésite entre un chauffage $27\%$, un cuisinière $25\%$ et un frigo $30\%$. On va donc présenter à l'utilisateur ces trois probabilités sur son interface et lui permettre de résoudre le doute. Dans les tables de la base de données cela s'apparentera à ceci :\\
\begin{table}[h]
\resizebox{\textwidth}{!}{%
\begin{tabular}{|c|c|c|c|c|c|c|c|c|c|c|c|c|c|c|}
\hline
\rowcolor[HTML]{EFEFEF} 
\begin{tabular}[c]{@{}c@{}}integer \\ Id type \\ appareils\end{tabular} & \begin{tabular}[c]{@{}c@{}}string \\ Nom type\\  appareils\end{tabular} & \begin{tabular}[c]{@{}c@{}}integer\\  POI1\end{tabular} & \begin{tabular}[c]{@{}c@{}}integer\\  POI2\end{tabular} & \begin{tabular}[c]{@{}c@{}}integer\\  POI3\end{tabular} & \begin{tabular}[c]{@{}c@{}}integer\\  POI4\end{tabular} & \begin{tabular}[c]{@{}c@{}}integer\\  POI5\end{tabular} & \begin{tabular}[c]{@{}c@{}}boolean\\  checked\end{tabular} & \begin{tabular}[c]{@{}c@{}}integer\\  Alias\end{tabular} & \begin{tabular}[c]{@{}c@{}}integer \\ Numero\\  type \\ appareil 1\end{tabular} & \begin{tabular}[c]{@{}c@{}}integer\\  Proba 1\end{tabular} & \begin{tabular}[c]{@{}c@{}}integer \\ Numero\\  type \\ appareil 2\end{tabular} & \begin{tabular}[c]{@{}c@{}}integer \\ Proba 2\end{tabular} & \begin{tabular}[c]{@{}c@{}}integer \\ Numero\\  type \\ appareil 3\end{tabular} & \begin{tabular}[c]{@{}c@{}}integer \\ Proba 3\end{tabular} \\ \hline
1 & chauffage & 10 & 15 & 20 & 49 & 200 & 1 & 0 &  &  &  &  &  &  \\ \hline
2 & frigo & 139 & 3829 & 1329 & 0284 & 1209 & 1 & 0 &  &  &  &  &  &  \\ \hline
3 & cuisinière & 17 & 8364 & 1893 & 1360 & 1361 & 1 & 0  &  &  &  &  &  &  \\ \hline
10000 & unknown & 1073 & 13 & 324 & 2486 & 1846 & 1 & -1 & 2 & 30 & 1 & 27 & 3 & 25 \\ \hline
 &  &  &  &  &  &  &  &  &  &  &  &  &  & 
\end{tabular}
}
\caption{Identification/interprétation des types d'appareils}
\end{table}

\begin{table}[h!]
\centering
    \begin{tabular}{|l|l|l|}
    \hline
    \rowcolor[HTML]{EFEFEF} 
    integer Id\_type\_appareils & integer Timestamp & boolean State \\ \hline
    1       & 1456787           & 0             \\ \hline
    1       & 1456788           & 1             \\ \hline
    2       & 1456790           & 1             \\ \hline
    1       & 1456795           & 0             \\ \hline
    3       & 1456798           & 1             \\ \hline
    10000       & 1456799          & 0             \\ \hline
                 &                   &              
    \end{tabular}
    \caption{États des appareils}
\end{table} 

L'utilisateur va résoudre le conflit en associant cet appareil inconnu à un chauffage. On va alors placer le champ alias de l'appareil inconnu de $-1$ à 1. Cet appareil inconnu sera désormais apparenté à un chauffage représenté par l'entier 1.\\

\begin{table}[h]
\resizebox{\textwidth}{!}{%
\begin{tabular}{|c|c|c|c|c|c|c|c|c|c|c|c|c|c|c|}
\hline
\rowcolor[HTML]{EFEFEF} 
\begin{tabular}[c]{@{}c@{}}integer \\ Id type \\ appareils\end{tabular} & \begin{tabular}[c]{@{}c@{}}string \\ Nom type\\  appareils\end{tabular} & \begin{tabular}[c]{@{}c@{}}integer\\  POI1\end{tabular} & \begin{tabular}[c]{@{}c@{}}integer\\  POI2\end{tabular} & \begin{tabular}[c]{@{}c@{}}integer\\  POI3\end{tabular} & \begin{tabular}[c]{@{}c@{}}integer\\  POI4\end{tabular} & \begin{tabular}[c]{@{}c@{}}integer\\  POI5\end{tabular} & \begin{tabular}[c]{@{}c@{}}boolean\\  checked\end{tabular} & \begin{tabular}[c]{@{}c@{}}integer\\  Alias\end{tabular} & \begin{tabular}[c]{@{}c@{}}integer \\ Numero\\  type \\ appareil 1\end{tabular} & \begin{tabular}[c]{@{}c@{}}integer\\  Proba 1\end{tabular} & \begin{tabular}[c]{@{}c@{}}integer \\ Numero\\  type \\ appareil 2\end{tabular} & \begin{tabular}[c]{@{}c@{}}integer \\ Proba 2\end{tabular} & \begin{tabular}[c]{@{}c@{}}integer \\ Numero\\  type \\ appareil 3\end{tabular} & \begin{tabular}[c]{@{}c@{}}integer \\ Proba 3\end{tabular} \\ \hline
1 & chauffage & 10 & 15 & 20 & 49 & 200 & 1 & 0 &  &  &  &  &  &  \\ \hline
2 & frigo & 139 & 3829 & 1329 & 0284 & 1209 & 1 & 0 &  &  &  &  &  &  \\ \hline
3 & cuisinière & 17 & 8364 & 1893 & 1360 & 1361 & 1 & 0  &  &  &  &  &  &  \\ \hline
10000 & unknown & 1073 & 13 & 324 & 2486 & 1846 & 1 & 1 & 2 & 30 & 1 & 27 & 3 & 25 \\ \hline
 &  &  &  &  &  &  &  &  &  &  &  &  &  & 
\end{tabular}
}
\caption{Identification/interprétation des types d'appareils}
\end{table}

\newpage
\subsubsection{Cas 2}
L'interprétation est encore indécise sur le type d'un appareil. En effet elle hésite entre un chauffage $27\%$, un cuisinière $29\%$ et un frigo $22\%$. On va donc à nouveau présenter à l'utilisateur ces trois probabilités sur son interface et lui permettre de résoudre le doute. Dans les tables de la base de données cela s'apparentera à ceci :\\

Voici les tables d'identification et d'états des appareils :
\begin{table}[h]
\resizebox{\textwidth}{!}{%
\begin{tabular}{|c|c|c|c|c|c|c|c|c|c|c|c|c|c|c|}
\hline
\rowcolor[HTML]{EFEFEF} 
\begin{tabular}[c]{@{}c@{}}integer \\ Id type \\ appareils\end{tabular} & \begin{tabular}[c]{@{}c@{}}string \\ Nom type\\  appareils\end{tabular} & \begin{tabular}[c]{@{}c@{}}integer\\  POI1\end{tabular} & \begin{tabular}[c]{@{}c@{}}integer\\  POI2\end{tabular} & \begin{tabular}[c]{@{}c@{}}integer\\  POI3\end{tabular} & \begin{tabular}[c]{@{}c@{}}integer\\  POI4\end{tabular} & \begin{tabular}[c]{@{}c@{}}integer\\  POI5\end{tabular} & \begin{tabular}[c]{@{}c@{}}boolean\\  checked\end{tabular} & \begin{tabular}[c]{@{}c@{}}integer\\  Alias\end{tabular} & \begin{tabular}[c]{@{}c@{}}integer \\ Numero\\  type \\ appareil 1\end{tabular} & \begin{tabular}[c]{@{}c@{}}integer\\  Proba 1\end{tabular} & \begin{tabular}[c]{@{}c@{}}integer \\ Numero\\  type \\ appareil 2\end{tabular} & \begin{tabular}[c]{@{}c@{}}integer \\ Proba 2\end{tabular} & \begin{tabular}[c]{@{}c@{}}integer \\ Numero\\  type \\ appareil 3\end{tabular} & \begin{tabular}[c]{@{}c@{}}integer \\ Proba 3\end{tabular} \\ \hline
1 & chauffage & 10 & 15 & 20 & 49 & 200 & 1 & 0 &  &  &  &  &  &  \\ \hline
2 & frigo & 139 & 3829 & 1329 & 0284 & 1209 & 1 & 0 &  &  &  &  &  &  \\ \hline
3 & cuisinière & 17 & 8364 & 1893 & 1360 & 1361 & 1 & 0  &  &  &  &  &  &  \\ \hline
10000 & unknown & 1073 & 13 & 324 & 2486 & 1846 & 1 & 1 & 2 & 30 & 1 & 27 & 3 & 25 \\ \hline
10001 & unknown & 328 & 2 & 2487 & 23 & 427 & 1 & -1 & 3 & 29 & 1 & 27 & 2 & 22 \\ \hline
 &  &  &  &  &  &  &  &  &  &  &  &  &  & 
\end{tabular}
}
\caption{Identification/interprétation des types d'appareils}
\end{table}

\begin{table}[h!]
\centering
    \begin{tabular}{|l|l|l|}
    \hline
    \rowcolor[HTML]{EFEFEF} 
    integer Id\_type\_appareils & integer Timestamp & boolean State \\ \hline
    1       & 1456787           & 0             \\ \hline
    1       & 1456788           & 1             \\ \hline
    2       & 1456790           & 1             \\ \hline
    1       & 1456795           & 0             \\ \hline
    3       & 1456798           & 1             \\ \hline
    10000       & 1456799          & 0             \\ \hline
    10001       & 1456800           & 1             \\ \hline
                 &                   &              
    \end{tabular}
    \caption{États des appareils}
\end{table} 
L'utilisateur va résoudre le conflit en associant cet appareil inconnu à un nouvel appareil non référencé : TV. On va alors placer le champ alias de l'appareil inconnu de $-1$ à $0$, changer le nom à TV.  
\begin{table}[h]
\resizebox{\textwidth}{!}{%
\begin{tabular}{|c|c|c|c|c|c|c|c|c|c|c|c|c|c|c|}
\hline
\rowcolor[HTML]{EFEFEF} 
\begin{tabular}[c]{@{}c@{}}integer \\ Id type \\ appareils\end{tabular} & \begin{tabular}[c]{@{}c@{}}string \\ Nom type\\  appareils\end{tabular} & \begin{tabular}[c]{@{}c@{}}integer\\  POI1\end{tabular} & \begin{tabular}[c]{@{}c@{}}integer\\  POI2\end{tabular} & \begin{tabular}[c]{@{}c@{}}integer\\  POI3\end{tabular} & \begin{tabular}[c]{@{}c@{}}integer\\  POI4\end{tabular} & \begin{tabular}[c]{@{}c@{}}integer\\  POI5\end{tabular} & \begin{tabular}[c]{@{}c@{}}boolean\\  checked\end{tabular} & \begin{tabular}[c]{@{}c@{}}integer\\  Alias\end{tabular} & \begin{tabular}[c]{@{}c@{}}integer \\ Numero\\  type \\ appareil 1\end{tabular} & \begin{tabular}[c]{@{}c@{}}integer\\  Proba 1\end{tabular} & \begin{tabular}[c]{@{}c@{}}integer \\ Numero\\  type \\ appareil 2\end{tabular} & \begin{tabular}[c]{@{}c@{}}integer \\ Proba 2\end{tabular} & \begin{tabular}[c]{@{}c@{}}integer \\ Numero\\  type \\ appareil 3\end{tabular} & \begin{tabular}[c]{@{}c@{}}integer \\ Proba 3\end{tabular} \\ \hline
1 & chauffage & 10 & 15 & 20 & 49 & 200 & 1 & 0 &  &  &  &  &  &  \\ \hline
2 & frigo & 139 & 3829 & 1329 & 0284 & 1209 & 1 & 0 &  &  &  &  &  &  \\ \hline
3 & cuisinière & 17 & 8364 & 1893 & 1360 & 1361 & 1 & 0  &  &  &  &  &  &  \\ \hline
10000 & unknown & 1073 & 13 & 324 & 2486 & 1846 & 1 & 1 & 2 & 30 & 1 & 27 & 3 & 25 \\ \hline
10001 & TV & 328 & 2 & 2487 & 23 & 427 & 1 & 0 & 3 & 29 & 1 & 27 & 2 & 22 \\ \hline
 &  &  &  &  &  &  &  &  &  &  &  &  &  & 
\end{tabular}
}
\caption{Identification/interprétation des types d'appareils}
\end{table}

\newpage
\section{Choix des technologies}
Nos contraintes pour les données sont les suivantes :
\begin{itemize}
\item La transmission des données est synchrone (les données vont être transmises successivement) ;
\item Les données sont triées, opérations de jointures sur les tables -> Besoin de base de données relationnelles ;
\item Les technologies doivent consommer le moins de ressources possibles pour fonctionner sur un dispositif tel qu'un Rasperry Pi ;
\end{itemize}   
\textbf{Propositions de technologies}:
\begin{itemize}
\item \textbf{Base de données SQLite : }Cette technologie légère et relationnelle peut être contenue dans un seul fichier. 
\item \textbf{Serveur Nginx ou lighttpd : }Ces serveurs demandent peu de resources et s'adaptent aux bases relationnelles.
\item \textbf{Langages PHP, J2EE, Ruby : }Ces langages natifs sont les plus adaptés aux serveurs énoncés précédemment.
\end{itemize}
\section{Organisation GLRT}
\begin{itemize}
\item Lien entre le récepteur/interprète et la base de données
\item Lien entre la base de données et le client web via le serveur
\item Client Web : Vues + Calculs
\end{itemize}



\end{document}