\documentclass[10pt,a4paper]{article}
\usepackage[utf8]{inputenc}
\usepackage[T1]{fontenc}
\usepackage[french]{babel}
\usepackage{enumitem}
%\frenchbsetup{StandardLists=true}
\usepackage{caption}
\usepackage{graphicx}
\usepackage{amsmath}
\usepackage{subfigure}
\usepackage{appendix}
\usepackage{listings}
\usepackage{multicol}
\usepackage{moreverb}
\usepackage{epsfig}
\usepackage{textcomp}
\usepackage{hyperref}
\usepackage{xcolor}
      %%%%%%%%%%%%%%%%  inclure la source %%%%%%%%%%%%%%%%%%%%
  \newcommand*\styleC{\fontsize{9}{10pt}\usefont{T1}{cmtt}{m}{n}\selectfont }
  \newcommand*\styleD{\fontsize{9}{10pt}\usefont{T1}{cmtt}{m}{n}\selectfont }

      \makeatletter
      % on fixe le langage utilisé
      \lstset{language=matlab, breaklines=true}
      \edef\Motscle{emph={\lst@keywords}}
      \expandafter\lstset\expandafter{%
       \Motscle}
      \makeatother


      \definecolor{Vvert}{rgb}{0.133,0.545,0.133}

      \lstset{emphstyle=\color{blue}, % les mots réservés de matlab en bleu
      basicstyle=\styleC,
      keywordstyle=\cmtt,
      commentstyle=\color{Vvert}\styleD, % commentaire en gris
      numberstyle=\tiny\color{red},
      numbers=left,
      numbersep=10pt,
      lineskip=0.7pt,
      showstringspaces=false}
      %  % inclure le fichier source
      \newcommand{\FSource}[1]{%
      \lstinputlisting[texcl=true]{#1}
      }

\setlength{\textwidth}{16cm} % Largeur du texte
\setlength{\textheight}{24cm} % Hauteur du texte
\setlength{\evensidemargin}{-0.2cm} % Taille des marges pour les pages paires
\setlength{\oddsidemargin}{-0.2cm} % Taille des marges pour les pages impaires
\setlength{\topmargin}{-1.5cm} % Taille de l'ent\^ete

\title{PR307 Smart-EcoBox : \\
Détection du chauffage dans le domaine temps-fréquence, approche par Spectrogramme.}
\author{Antoine Bouchain <abouchain@enseirb-matmeca.fr> \\}


\begin{document}
\maketitle

\begin{figure}[ht]
\begin{center}
\noindent \includegraphics[scale = 0.25]{../../Logo_ENSEIRB.pdf}
\end{center}
\end{figure}
\begin{figure}[ht]
\begin{center}
\noindent \includegraphics[scale = 0.25]{../../smartee_logo.pdf}
\end{center}
\end{figure}


\newpage

\tableofcontents

\newpage
\begin{abstract}
Ce rapport a pour but d'expliquer la détection d'un chauffage par analyse temps-fréquence, à l'aide d'un spectrogramme de la courbe de charge générale d'une maison. Ces travaux sont issus de la thèse de M. El Guedri intitulée "Détection, classification et estimation des usages dans les secteurs résidentiel et tertiaire." 
\end{abstract}

\section{Interprétation du Spectrogramme}
Le spectrogramme d'une courbe de charge présentant un chauffage laisse apparaitre des raies horizontales et équidistantes significatives, comme illustré figure 1 et 2. Ces raies couvrent l'ensemble du spectre, ce qui n'est pas le cas pour les autres appareils.

\begin{figure}[!h]
\begin{center}
\includegraphics[scale = 0.4]{./../../../fig/cdc_chauffage.jpg}
\caption{figure 1 : Courbe de Charge du chauffage}
\label{Figure}
\end{center}
\end{figure}

\begin{figure}[!h]
\begin{center}
\includegraphics[scale = 0.4]{./../../../fig/spectro_chauffage.jpg}
\caption{figure 2 : Spectrogramme du chauffage}
\label{Figure}
\end{center}
\end{figure}


\section{Segmentation du Spectrogramme}
 On souhaite réaliser une segmentation en deux régions : une avec les motifs spectraux du chauffage et une sans ces motifs.
 
 
\section{Seuillage de la carte temps-fréquence}


\section{Extraction des motifs fréquents}


\section{Classification des motifs fréquents}


\section{Détermination du domaine temporel}





\end{document}


%\begin{figure}[!h]
%\begin{center}
%\includegraphics[scale = 0.4]{./fig/fig1.jpg}
%\caption{figure}
%\label{Figure}
%\end{center}
%\end{figure}
