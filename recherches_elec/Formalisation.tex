\documentclass{beamer}
%
% Choose how your presentation looks.
%
% For more themes, color themes and font themes, see:
% http://deic.uab.es/~iblanes/beamer_gallery/index_by_theme.html
%
\mode<presentation>
{
  \usetheme{Darmstadt}      % or try Darmstadt, Madrid, Warsaw, ...
  \usecolortheme{default} % or try albatross, beaver, crane, ...
  \usefonttheme{default}  % or try serif, structurebold, ...
  \setbeamertemplate{navigation symbols}{}
  \setbeamertemplate{caption}[numbered]
} 

\usepackage[english]{babel}
\usepackage[utf8x]{inputenc}

\title[Your Short Title]{Formalisation de la détection d'appareils électroménagers}
\author{Antoine Bouchain}
\date{23 Décembre 2014}

\begin{document}

\begin{frame}
  \titlepage
\end{frame}

% Uncomment these lines for an automatically generated outline.
%\begin{frame}{Outline}
%  \tableofcontents
%\end{frame}

\begin{frame}{Plan}

\begin{itemize}
  \item Détection de fonctionnements périodiques On/Off
  \begin{itemize}
  	\item Détection du chauffage par taux d'accroissement.
  	\item Détection du chauffage par spectrogramme.
  \end{itemize}
  \item Détection de fonctionnements apériodiques On/Off.
  \begin{itemize}
  	\item Détection du lave-linge.
  	\item Détection du lave-vaisselle.
  	\item Détection du sèche-linge. 
  \end{itemize}
  \item Détection de fonctionnements ponctuels On/Off.
  \begin{itemize}
  	\item Détection d'appareils de cuisson (plaque, four).
  	\item Détection d'un micro-onde.
  \end{itemize}
\end{itemize}

\begin{block}{Note}
Etude basée sur la thèse de M. El Guedri : "Détection, classification et estimation des usages dans les secteurs résidentiels et tertiaires."
\end{block}

\end{frame}

\section{Fonctionnement périodique On/Off}

\begin{frame}{Fonctionnement périodique On/Off}
\end{frame}

\subsection{Détection par taux d'accroissement}

\begin{frame}{Détection par taux d'accroissement}
On définit le taux d'accroissement comme : $x_{ta}(t) = x(t+1) - x(t)$, $t \le N$, si $N$ est la longueur du signal considéré $x$.
\begin{figure}[!h]
\begin{center}
\includegraphics[scale = 0.32]{./../../../fig/taux_acc.png}
\caption{Courbe de charge et Taux d'accroissement}
\label{Figure1}
\end{center}
\end{figure}
\end{frame}

\begin{frame}
La figure précédente montre la présence de 3 utilisations de chauffage. On suppose qu'un chauffage a une consommation électrique d'au moins 750 W. De plus, un chauffage électrique a une courbe de charge en forme de créneaux, on va donc chercher à détecter ces créneaux. Le taux d'accroissement d'un créneau est une alternance de points positifs puis négatifs. Tous ces points sont de longueur variable.
\end{frame}

\begin{frame}{Algorithme}
Etape 0 :\\
Supposons qu'en $t = t_0$, la courbe de charge, $x$ dépasse 750 W. On va chercher sur les $c$ prochains points si $x(t), t_0 \le t \le t_0 + c $ devient inférieur à -750W.\\
\vspace{0.5 cm}
Etape 1:\\
\begin{itemize}
	\item S'il existe $t_1$ tel que $x(t_1) \le$ -750W alors on va chercher sur les $c$ points suivant si $x(t), t_1 \le t \le t_1 + c $ devient supérieur à 750W. On repart alors à l'étape 0 et ainsi de suite ...
	\item Sinon l'algorithme s'arrête car il n'y a plus de créneau, le chauffage s'est arrêté.
\end{itemize}
\end{frame}

\begin{frame}{Résultats de simulation}
Nous avons également ajouté à l'algorithme de détection de chauffage par taux d'accroissement une simple détection d'appareil par franchissement de seuils fixés à 100 et 500W.
\begin{figure}[!h]
\begin{center}
\includegraphics[scale = 0.32]{./../../../fig/detection.png}
\caption{Message de détection des appareils}
\label{Figure2}
\end{center}
\end{figure}
\end{frame}

\begin{frame}
\begin{figure}[!h]
\begin{center}
\includegraphics[scale = 0.32]{./../../../fig/chauff_simule.png}
\caption{Courbe de charge synthétique d'un chauffage}
\label{Figure3}
\end{center}
\end{figure}
\end{frame}

\begin{frame}
\begin{figure}[!h]
\begin{center}
\includegraphics[scale = 0.32]{./../../../fig/chauff_simule2.png}
\caption{Estimation de la consommation d'un chauffage sur signal synthétique}
\label{Figure4}
\end{center}
\end{figure}
\end{frame}

\begin{frame}
\begin{figure}[!h]
\begin{center}
\includegraphics[scale = 0.32]{./../../../fig/conso_chauffage_2.png}
\caption{Estimation de la consommation d'un chauffage}
\label{Figure5}
\end{center}
\end{figure}
\end{frame}

\subsection{Détection par spectrogramme}

\begin{frame}{Détection par spectrogramme}
\begin{itemize}
	\item Le spectrogramme d'une courbe de charge présentant un chauffage laisse apparaitre des raies horizontales et équidistantes significatives, comme illustré figure 1 et 2.
	\item Ces raies couvrent l'ensemble du spectre, ce qui n'est pas le cas pour les autres appareils.
\end{itemize}
\end{frame}

\begin{frame}

\begin{figure}[!h]
\begin{center}
\includegraphics[scale = 0.18]{./../../../fig/cdc_chauffage.jpg}
\caption{Courbe de Charge du chauffage}
\label{Figure6}
\end{center}
\end{figure}

\begin{figure}[!h]
\begin{center}
\includegraphics[scale = 0.18]{./../../../fig/spectro_chauffage.jpg}
\caption{Spectrogramme du chauffage}
\label{Figure7}
\end{center}
\end{figure}

\end{frame}

\begin{frame}{Seuillage de la carte temps-fréquence}
Considérons une portion de signal présentant 3 utilisations du chauffage : à 23h20, 1h et 2h30.
\begin{figure}[!h]
\begin{center}
\includegraphics[scale = 0.3]{./../../../fig/cdc_spectro_chauffage.jpg}
\caption{Courbe de charge et spectrogramme considérés}
\label{Figure8}
\end{center}
\end{figure}
\end{frame}

\begin{frame}
On effectue un seuillage du spectrogramme $S(x,y)$ pour différencier le signal chauffage du bruit de fond. On suppose 2 hypothèses :
\begin{itemize}
	\item \textit{$H_0$ : Signal aléatoire stationnaire, de puissance moyenne finie, éventuellement augmenté d'un signal déterministe, sa densité de probabilité marginale $p_0$ est connue mais de paramètres inconnus. (Signal de bruit)}
	\item \textit{$H_1$ : Signal stationnaire par morceaux de densité de probabilité inconnue. (Signal de chauffage)}
\end{itemize}
\end{frame}

\begin{frame}
Le problème peut se réécrire comme suit : \\
Si $\rho_y[k,\nu] \ge \eta_{P_{fa}}[\nu]$ alors on retient $H_0$.
Avec $\eta_{P_{fa}}[\nu]$ le seuil de détection fixé pour une fréquence donnée en fonction de la probabilité de fausse alarme choisie. \\
On l'obtient par l'équation :

\begin{equation}
\int_{\eta_{P_{fa}}[\nu]}^\infty p_0(u)\, \mathrm du
\label{eq1}
\end{equation}

Pour résoudre cette équation il faut tenir compte du modèle statistique du spectrogramme :
\begin{itemize}
	\item \textit{$H_0^*$ : Signal aléatoire stationnaire b, gaussien centré de variance $\sigma^2$, éventuellement augmenté d'un signal déterministe d.}
	\item \textit{$H_1^*$ : rejet de $H_0^*$.}
\end{itemize}
\end{frame}

\begin{frame}
La fenêtre d'analyse utilisée est rectangulaire.\\
Alors d'après le théorème central limite, chaque coefficient du spectrogramme du signal $b$ suit une loi du $\chi_2$ à $l$ degrés de libertés (DSP : $|$ $|^2$). Par conséquent, chaque coefficient du spectrogramme suit une loi Gamma $\Gamma(l/2,\sigma^2/2,0).$ \\
\vspace{0.25cm}
Un signal déterministe vient s'additionner au bruit blanc gaussien observé, alors les coefficients de la TFCT suivent les lois du $\chi_2$ décentrés du même degré de liberté $l$ et dont les paramètres de décentrage sont définis par la relation suivante :

\begin{equation}
\delta = l \frac{\rho_d[k,\nu]}{\rho_b[k,\nu]}
\label{eq2}
\end{equation}
où $\rho_d[k,\nu]$ et $\rho_b[k,\nu]$ sont respectivement les coefficients du spectrogramme du signal $d$ et $b$.

Le spectrogramme est donc modélisé par un mélange de lois du $\chi_2$ centrés et décentrés. Les paramètres des lois décentrés sont fonction du SNR.
\end{frame}

\begin{frame}{Algorithme de détection de non stationnarité du spectrograme}
Le test de détection dans le plan temps-fréquence repose sur la connaissance de la moyenne des éléments de $L_{H0}[\nu]$ que l'on cherche. C'est pourquoi la construction des deux espaces d'observation $L_{H0}$ et $L_{H1}$ est réalisée itérativement. 
\end{frame}

\begin{frame}
\begin{block}{Algorithme}
Pour chaque fréquence, $\nu$, $L_{H0}[\nu]$ est initialisé par les $p$ \% coefficients de plus faible intensité, $p$ ici est de 80 \%. En effet, les amplitudes des coefficients ont de plus faibles intensités sous $H_0$ que sous $H_1$. \\
A chaque itération, le seuil de détection $\eta_{P_{fa}}[\nu]$ est actualisé en fonction de la moyenne de l'ensemble des observations de $L_{H0}[\nu]$ à l'itération précédente et les deux espaces d'observations sont actualisés. \\
Le critère d'arrêt est la stabilité des deux ensembles recherchés. On obtient alors une carte temps-fréquence où les éléments de $L_{H1}$ sont à 1 et ceux de $L_{H0}$ sont à -10.
\end{block}
\end{frame}

\begin{frame}
\begin{figure}[!h]
\begin{center}
\includegraphics[scale = 0.3]{./../../../fig/spectro_seuil.jpg}
\caption{Spectrogramme du chauffage avec seuil}
\label{Figure9}
\end{center}
\end{figure}
\end{frame}

\begin{frame}{Extraction des motifs fréquents}
L'idée ensuite est de réaliser une recherche de motifs rectangulaires.

\begin{figure}[!h]
\begin{center}
\includegraphics[scale = 0.35]{./../../../fig/spectro_seuil_2.jpg}
\caption{Spectrogramme avec motifs repérés}
\label{Figure10}
\end{center}
\end{figure}
\end{frame}

\begin{frame}
\begin{itemize}
	\item Ces motifs seront recherchés à des fréquences spécifiques. 			\item Nous en choisissons 20 idéalement reparties entre 0.025 et 0.25.
	\item Pour chaque fréquence $f_i, 1\le i \le 20$, on compte le nombre d'occurrence de 1 sur une fenêtre temporelle donnée (de longueur $N_f = 256$ points et de largeur 2).
	\item Si ce nombre d'occurrence apparait sur toute toute la durée de la fenêtre, c'est que nous avons affaire à une raie qui pourrait être un chauffage.
	\item On obtient alors une séries de rectangles qui caractérisent la présence ou non des raies spécifiques du chauffage.
\end{itemize}
\end{frame}

\begin{frame}
\begin{figure}[!h]
\begin{center}
\includegraphics[scale = 0.4]{./../../../fig/spectro_motif.jpg}
\caption{Spectrogramme avec motifs rectangulaires}
\label{Figure11}
\end{center}
\end{figure}
\end{frame}

\begin{frame}{Détermination du domaine temporel}
\begin{itemize}
	\item Nous faisons alors une somme de chaque rectangle.
	\item Si la somme est strictement inférieure au nombre de fréquences d'intérêt, 20, il n'y a pas de chauffage en marche.
	\item Sinon nous supposons le chauffage en marche et affectons une valeur arbitraire de 750 W à la courbe estimée de consommation.
	\item Nous effectuons également un recadrement temporel pour correspondre à la courbe de charge initiale de la figure 3.
\end{itemize}
\end{frame}

\begin{frame}
\begin{figure}[!h]
\begin{center}
\includegraphics[scale = 0.4]{./../../../fig/conso_chauffage.jpg}
\caption{Estimation de la consommation du chauffage.}
\label{Figure12}
\end{center}
\end{figure}
\end{frame}

\subsection{Conclusion sur la détection du chauffage}
\begin{frame}{Conclusion sur la détection du chauffage}
\begin{itemize}
	\item La méthode basée sur le taux d'accroissement est plus simple à mettre en œuvre et est plus précise en terme d'estimation de durée d'utilisation que celle basée sur le spectrogramme.
	\item Dans les deux cas, nous ne pouvons détecter l'utilisation de deux chauffage en même temps.
\end{itemize}
\end{frame}

\section{Fonctionnement apériodique On/Off}

\begin{frame}{Fonctionnement apériodique On/Off}
\end{frame}

\subsection{Détection d'un lave-linge}
\begin{frame}{Détection d'un lave-linge}
Lave-linge : tambour, résistance de chauffage et pompe.\\
\vspace{0.25cm}
Le signal observé est un mélange de deux signaux élémentaires : des régulations rapides (de l'ordre d'une minute) et de faible amplitude ($\simeq$ 100 W) et des créneaux d'amplitude élevée ($\ge$ 2000 W) qui correspondent aux chauffage.
\end{frame}

\begin{frame}
\begin{figure}[!h]
\begin{center}
\includegraphics[scale = 0.5]{./../../../fig/lavelinge.png}
\caption{Courbe de charge d'un lave-linge}
\label{Figure13}
\end{center}
\end{figure}
\end{frame}

\subsection{Détection d'un lave-vaisselle}
\begin{frame}{Détection d'un lave-vaisselle}
Lave-vaisselle : résistance de chauffage et pompe.\\
\vspace{0.25cm}
Le signal observé est un mélange de deux familles de signaux : un signal de faible amplitude ($\simeq$ 200 W) qui correspond à la puissance appelée par le moteur et un signal de forte amplitude ($\ge$ 2000 W) qui correspond à la puissance appelée par la résistance du chauffage de l'eau.
\end{frame}

\begin{frame}
\begin{figure}[!h]
\begin{center}
\includegraphics[scale = 0.4]{./../../../fig/lavevaisselle.png}
\caption{Courbe de charge synthétique d'un lave-vaisselle}
\label{Figure14}
\end{center}
\end{figure}
\end{frame}

\begin{frame}
\begin{figure}[!h]
\begin{center}
\includegraphics[scale = 0.4]{./../../../fig/lavevaisselle_simule.png}
\caption{Courbe de charge d'un lave-vaisselle}
\label{Figure15}
\end{center}
\end{figure}
\end{frame}

\begin{frame}
Nous réalisons de la même manière que pour le chauffage, la détection simultanée de créneau d'amplitude  200 et 2000W. Si nous détectons plus de 2 créneaux de chaque amplitude différente alors nous considérons qu'un lave-vaisselle est effectivement en marche.
\end{frame}

\begin{frame}
\begin{figure}[!h]
\begin{center}
\includegraphics[scale = 0.4]{./../../../fig/lavevaisselle_simule2.png}
\caption{Estimation de la consommation d'un lave-vaisselle sur un signal synthétique}
\label{Figure16}
\end{center}
\end{figure}
\end{frame}

\subsection{Détection d'un sèche-linge}
\begin{frame}{Détection d'un sèche-linge}
Sèche-linge : résistance de chauffage, tambour et pompe.\\
\vspace{0.25cm}
Le signal observé est un mélange d'un créneau (A1 = 200W) et d'un créneau quasi-périodique (A2 = 100 W), qui correspondent respectivement à la résistance et au moteur. Sur la fin, on observe un créneau quasi-périodique, de faible amplitude, qui correspondrait à la puissance appelée par la pompe.
\end{frame}

\begin{frame}
\begin{figure}[!h]
\begin{center}
\includegraphics[scale = 0.32]{./../../../fig/sechelinge.png}
\caption{Courbe de charge d'un sèche-linge}
\label{Figure17}
\end{center}
\end{figure}
\end{frame}

\section{Fonctionnement ponctuel On/Off}

\begin{frame}{Fonctionnement ponctuel On/Off}
\end{frame}

\subsection{Appareils de cuisson}
\begin{frame}{Appareils de cuisson : plaque et four}
La plaque de cuisson, elle, présente un caractère de décroissance exponentielle.\\
Le four se caractérise par un simple créneau, puisque c'est une résistance.
\begin{figure}[!h]
\begin{center}
\includegraphics[scale = 0.32]{./../../../fig/four.jpg}
\caption{Courbe de charge d'une plaque de cuisson et d'un four}
\label{Figure18}
\end{center}
\end{figure}
\end{frame}

\subsection{Micro-onde}
\begin{frame}{Micro-onde}
Le four micro-onde se caractérise par une résistance et un moteur. La durée de fonctionnement est proche de celle d'un chauffage. On a néanmoins, la présence d'un pic de puissance avant le créneau du chauffage et du moteur.
\begin{figure}[!h]
\begin{center}
\includegraphics[scale = 0.3]{./../../../fig/microonde.jpg}
\caption{Courbe de charge d'un micro-onde.}
\label{Figure19}
\end{center}
\end{figure}
\end{frame}

%\begin{frame}
%Some text goes here
%\begin{figure}[!h]
%\begin{center}
%\includegraphics[scale = 0.4]{./../../../fig/microonde.jpg}
%\caption{Courbe de charge d'un micro-onde.}
%\label{Figure1}
%\end{center}
%\end{figure}
%\end{frame}

\end{document}
