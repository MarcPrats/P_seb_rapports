\documentclass{beamer}
%
% Choose how your presentation looks.
%
% For more themes, color themes and font themes, see:
% http://deic.uab.es/~iblanes/beamer_gallery/index_by_theme.html
%
\mode<presentation>
{
  \usetheme{Darmstadt}      % or try Darmstadt, Madrid, Warsaw, ...
  \usecolortheme{default} % or try albatross, beaver, crane, ...
  \usefonttheme{default}  % or try serif, structurebold, ...
  \setbeamertemplate{navigation symbols}{}
  \setbeamertemplate{caption}[numbered]
} 

\usepackage[english]{babel}
\usepackage[utf8x]{inputenc}

\title[Your Short Title]{Présentation analyse macroscopique}
\author{Antoine Bouchain}
\date{24 Novembre 2014}

\begin{document}

\begin{frame}
  \titlepage
\end{frame}

% Uncomment these lines for an automatically generated outline.
%\begin{frame}{Outline}
%  \tableofcontents
%\end{frame}

\section{Introduction}

\begin{frame}{Plan}

\begin{itemize}
  \item Détection d'un chauffage dans une courbe de charge générale.
  \item Retour sur le seuillage.
  \item Classification des appareils d'une maison.
\end{itemize}

\vskip 1cm

\begin{block}{Note}
Etude basée sur la thèse de M. El Guedri : "Détection, classification et estimation des usages dans les secteurs résidentiels et tertiaires."
\end{block}

\end{frame}

\section{Détection d'un chauffage}

\begin{frame}{Détection d'un chauffage}
\end{frame}

\subsection{Interprétation du Spectrogramme}

\begin{frame}{Interprétation du Spectrogramme}
\begin{itemize}
	\item Le spectrogramme d'une courbe de charge présentant un chauffage laisse apparaitre des raies horizontales et équidistantes significatives, comme illustré figure 1 et 2.
	\item Ces raies couvrent l'ensemble du spectre, ce qui n'est pas le cas pour les autres appareils.
\end{itemize}
\end{frame}

\begin{frame}

\begin{figure}[!h]
\begin{center}
\includegraphics[scale = 0.2]{./../../../fig/cdc_chauffage.jpg}
\caption{Courbe de Charge du chauffage}
\label{Figure1}
\end{center}
\end{figure}

\begin{figure}[!h]
\begin{center}
\includegraphics[scale = 0.2]{./../../../fig/spectro_chauffage.jpg}
\caption{Spectrogramme du chauffage}
\label{Figure2}
\end{center}
\end{figure}

\end{frame}

\subsection{Seuillage de la carte temps-fréquence}

\begin{frame}{Seuillage de la carte temps-fréquence}
Considérons une portion de signal présentant 3 utilisations du chauffage : à 23h20, 1h et 2h30.

\begin{figure}[!h]
\begin{center}
\includegraphics[scale = 0.3]{./../../../fig/cdc_spectro_chauffage.jpg}
\caption{Courbe de charge et spectrogramme considérés}
\label{Figure3}
\end{center}
\end{figure}
\end{frame}

\begin{frame}
Chaque point $S(x,y)$ du spectrogramme de la figure 2 est seuillé, arbitrairement, de la manière suivante :
\begin{itemize}
	\item Si $S(x,y) \le 40 dB$ alors $S(x,y) = -10 dB$
	\item Sinon $S(x,y) = 1 dB$
\end{itemize}

\begin{figure}[!h]
\begin{center}
\includegraphics[scale = 0.3]{./../../../fig/spectro_seuil.jpg}
\caption{Spectrogramme du chauffage avec seuil à 40 dB}
\label{Figure4}
\end{center}
\end{figure}
\end{frame}

\subsection{Extraction des motifs fréquents}
\begin{frame}{Extraction des motifs fréquents}
L'idée ensuite est de réaliser une recherche de motifs rectangulaires.

\begin{figure}[!h]
\begin{center}
\includegraphics[scale = 0.4]{./../../../fig/spectro_seuil_2.jpg}
\caption{Spectrogramme avec motifs repérés}
\label{Figure5}
\end{center}
\end{figure}
\end{frame}

\begin{frame}
\begin{itemize}
	\item Ces motifs seront recherchés à des fréquences spécifiques. 			\item Nous en choisissons 20 idéalement reparties entre 0.025 et 0.25.
	\item Pour chaque fréquence $f_i, 1\le i \le 20$, on compte le nombre d'occurrence de 1 sur une fenêtre temporelle donnée (de longueur $N_f = 256$ points et de largeur 2).
	\item Si ce nombre d'occurrence apparait sur toute toute la durée de la fenêtre, c'est que nous avons affaire à une raie qui pourrait être un chauffage.
	\item On obtient alors une séries de rectangles qui caractérisent la présence ou non des raies spécifiques du chauffage.
\end{itemize}
\end{frame}

\begin{frame}
\begin{figure}[!h]
\begin{center}
\includegraphics[scale = 0.4]{./../../../fig/spectro_motifs.jpg}
\caption{Spectrogramme avec motifs rectangulaires}
\label{Figure6}
\end{center}
\end{figure}
\end{frame}

\subsection{Détermination du domaine temporel}
\begin{frame}{Détermination du domaine temporel}
\begin{itemize}
	\item Nous faisons alors une somme de chaque rectangle.
	\item Si la somme est strictement inférieure au nombre de fréquences d'intérêt, 20, il n'y a pas de chauffage en marche.
	\item Sinon nous supposons le chauffage en marche et affectons une valeur arbitraire de 750 W à la courbe estimée de consommation.
	\item Nous effectuons également un recadrement temporel pour correspondre à la courbe de charge initiale de la figure 3.
\end{itemize}
\end{frame}

\begin{frame}
\begin{figure}[!h]
\begin{center}
\includegraphics[scale = 0.4]{./../../../fig/conso_chauffage.jpg}
\caption{Estimation de la consommation du chauffage.}
\label{Figure7}
\end{center}
\end{figure}
\end{frame}

\subsection{Conclusion sur la détection du chauffage}
\begin{frame}{Conclusion}
\begin{itemize}
	\item La méthode présentée nécessite de nombreux calculs pour un résultat non optimal. Nous voyons clairement que la courbe d'estimation du chauffage considère une utilisation plus longue de celui-ci, par rapport à la véritable consommation.
	\item Nous allons essayer de nous intéresser maintenant à une méthode d'avantage basée sur l'aspect temporel en s'intéressant aux différentes valeurs d'oscillation de la courbe de charge.
\end{itemize}
\end{frame}

\section{Retour sur le seuillage}
\begin{frame}{Retour sur le seuillage}
\end{frame}
\subsection{Segmentation du spectrogramme}
\begin{frame}{Principe}
Dans la thèse de M. El Guedri, le seuillage du spectrogramme $\rho_y[k,\nu]$ est réalisé en 2 régions $L_{H_0}$, motifs spectraux du chauffage, et $L_{H_1}$, bruit de fond de la courbe de charge. Son détecteur de motifs s'appuie sur les variations du moment d'ordre 2 du signal dans le domaine-temps fréquence.

On suppose 2 hypothèses :
\begin{itemize}
	\item \textit{$H_0$ : Signal aléatoire stationnaire, de puissance moyenne finie, éventuellement augmenté d'un signal déterministe, sa densité de probabilité marginale $p_0$ est connue mais de paramètres inconnus.}
	\item \textit{$H_1$ : Signal stationnaire par morceaux de densité de probabilité inconnue.}
\end{itemize}
\end{frame}

\begin{frame}
Le problème peut se réécrire comme suit : \\
Si $\rho_y[k,\nu] \ge \eta_{P_{fa}}[\nu]$ alors on retient $H_0$.
Avec $\eta_{P_{fa}}[\nu]$ le seuil de détection fixé pour une fréquence donnée en fonction de la probabilité de fausse alarme choisie. \\
On l'obtient par l'équation :

\begin{equation}
\int_{\eta_{P_{fa}}[\nu]}^\infty p_0(u)\, \mathrm du
\label{eq1}
\end{equation}

Pour résoudre cette équation il faut tenir compte du modèle statistique du spectrogramme :
\begin{itemize}
	\item \textit{$H_0^*$ : Signal aléatoire stationnaire b, gaussien centré de variance $\sigma^2$, éventuellement augmenté d'un signal déterministe d.}
	\item \textit{$H_1^*$ : rejet de $H_0^*$.}
\end{itemize}
\end{frame}

\begin{frame}
La fenêtre d'analyse utilisée est rectangulaire.\\
Alors d'après le théorème central limite, chaque coefficient du spectrogramme du signal $d$ suit une loi du $\chi_2$ à $l$ degrés de libertés.\\
Par conséquent, chaque coefficient du spectrogramme suit une loi Gamma $\Gamma(l/2,\sigma^2/2,0).$ \\
\vspace{0.25cm}
Un signal déterministe vient s'additionner au bruit blanc gaussien observé, alors les coefficients de la TFCT suivent les lois du $\chi_2$ décentrés du même degré de liberté $l$ et dont les paramètres de décentrage sont définis par la relation suivante :

\begin{equation}
\delta = l \frac{\rho_d[k,\nu]}{\rho_b[k,\nu]}
\label{eq2}
\end{equation}
où $\rho_d[k,\nu]$ et $\rho_b[k,\nu]$ sont respectivement les coefficients du spectroframme du signal $d$ et $b$.

Le spectrogramme est donc modélisé par un mélange de lois du $\chi_2$ centrés et décentrés. Les paramètres des lois décentrés sont fonction du SNR.
\end{frame}

\subsection{Algorithme de détection de non stationnarité du spectrograme}
\begin{frame}{Algorithme}
Le test de détection dans le plan temps-fréquence repose sur la connaissance de la moyenne des éléments de $L_{H0}[\nu]$ que l'on cherche. C'est pourquoi la construction des deux espaces d'observation $L_{H0}$ et $L_{H1}$ est réalisée itérativement. 
\end{frame}

\begin{frame}
\begin{block}{Algorithme}
Pour chaque fréquence, $\nu$, $L_{H0}[\nu]$ est initialisé par les $p$ \% coefficients de plus faible intensité, $p$ ici est de 80 \%. En effet, les amplitudes des coefficients ont de plus faibles intensités sous $H_0$ que sous $H_1$. \\
A chaque itération, le seuil de détection $\eta_{P_{fa}}[\nu]$ est actualisé en fonction de la moyenne de l'ensemble des observations de $L_{H0}[\nu]$ à l'itération précédente et les deux espaces d'observations sont actualisés. \\
Le critère d'arrêt est la stabilité des deux ensembles recherchés. On obtient alors une carte temps-fréquence où les éléments de $L_{H1}$ sont colorés et ceux de $L_{H0}$ sont à 0.
\end{block}
\end{frame}

\begin{frame}
\begin{figure}[!h]
\begin{center}
\includegraphics[scale = 0.4]{./../../../fig/algo_seuil.jpg}
\caption{Schéma général de la détection du chauffage dans le domaine temps-fréquence.}
\label{Figure8}
\end{center}
\end{figure}
\end{frame}

\section{Classification des appareils d'une maison}
\begin{frame}{Classification des appareils d'une maison}
\end{frame}

\subsection{Gros appareils d'une maison}
\begin{frame}{Gros appareils d'une maison}
Lave-linge : tambour, résistance de chauffage et pompe.\\
\vspace{0.25cm}
Le signal observé est un mélange de deux signaux élémentaires : des régulations rapides (de l'ordre d'une minute) et de faible amplitude ($\simeq$ 100 W) et des créneaux d'amplitude élevée ($\ge$ 2000 W) qui correspondent aux chauffage.
\begin{figure}[!h]
\begin{center}
\includegraphics[scale = 0.25]{./../../../fig/lavelinge.jpg}
\caption{Courbe de charge d'un lave-linge}
\label{Figure9}
\end{center}
\end{figure}
\end{frame}

\begin{frame}
Lave-vaisselle : résistance de chauffage et pompe.\\
\vspace{0.25cm}
Le signal observé est un mélange de deux familles de signaux : un signal de faible amplitude ($\simeq$ 200 W) qui correspond à la puissance appelée par le moteur et un signal de forte amplitude ($\ge$ 2000 W) qui correspond à la puissance appelée par la résistance du chauffage de l'eau.
\begin{figure}[!h]
\begin{center}
\includegraphics[scale = 0.4]{./../../../fig/lavevaisselle.jpg}
\caption{Courbe de charge d'un lave-vaisselle}
\label{Figure10}
\end{center}
\end{figure}
\end{frame}

\begin{frame}
Sèche-linge : résistance de chauffage, tambour et pompe.\\
\vspace{0.25cm}
Le signal observé est un mélange d'un créneau (A1 = 200W) et d'un créneau quasi-périodique (A2 = 100 W), qui correspondent respectivement à la résistance et au moteur. Sur la fin, on observe un créneau quasi-périodique, de faible amplitude, qui correspondrait à la puissance appelée par la pompe.
\begin{figure}[!h]
\begin{center}
\includegraphics[scale = 0.35]{./../../../fig/sechelinge.jpg}
\caption{Courbe de charge d'un sèche-linge}
\label{Figure11}
\end{center}
\end{figure}
\end{frame}

\subsection{Les appareils de cuisson}
\begin{frame}{Les appareils de cuisson}
La plaque de cuisson, elle, présente un caractère de décroissance exponentielle.\\
Le four se caractérise par un simple créneau, puisque c'est une résistance.
\begin{figure}[!h]
\begin{center}
\includegraphics[scale = 0.4]{./../../../fig/four.jpg}
\caption{Courbe de charge d'une plaque de cuisson et d'un four}
\label{Figure12}
\end{center}
\end{figure}
\end{frame}

\begin{frame}
Le four micro-onde se caractérise par une résistance et un moteur. La durée de fonctionnement est proche de celle d'un chauffage. On a néanmoins, la présence d'un pic de puissance avant le créneau du chauffage et du moteur.
\begin{figure}[!h]
\begin{center}
\includegraphics[scale = 0.4]{./../../../fig/microonde.jpg}
\caption{Courbe de charge d'un micro-onde.}
\label{Figure12}
\end{center}
\end{figure}
\end{frame}

\end{document}
