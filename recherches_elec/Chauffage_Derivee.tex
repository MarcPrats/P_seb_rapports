\documentclass{beamer}
%
% Choose how your presentation looks.
%
% For more themes, color themes and font themes, see:
% http://deic.uab.es/~iblanes/beamer_gallery/index_by_theme.html
%
\mode<presentation>
{
  \usetheme{Darmstadt}      % or try Darmstadt, Madrid, Warsaw, ...
  \usecolortheme{default} % or try albatross, beaver, crane, ...
  \usefonttheme{default}  % or try serif, structurebold, ...
  \setbeamertemplate{navigation symbols}{}
  \setbeamertemplate{caption}[numbered]
} 

\usepackage[english]{babel}
\usepackage[utf8x]{inputenc}

\title[Your Short Title]{Présentation analyse macroscopique par taux d'accroissement}
\author{Antoine Bouchain}
\date{15 Décembre 2014}

\begin{document}

\begin{frame}
  \titlepage
\end{frame}

% Uncomment these lines for an automatically generated outline.
%\begin{frame}{Outline}
%  \tableofcontents
%\end{frame}

\section{Introduction}

\begin{frame}{Plan}

\begin{itemize}
  \item Détection d'un chauffage dans une courbe de charge générale.
  \item Avantages et inconvénients de la méthode.
\end{itemize}

\vskip 1cm

\end{frame}

\section{Détection d'un chauffage}

\begin{frame}{Détection d'un chauffage}
\end{frame}

\subsection{Interprétation du taux d'accroissement}

\begin{frame}{Interprétation du taux d'accroissement}
On définit le taux d'accroissement comme : $x_{ta}(t) = x(t+1) - x(t)$, $t \le N$, si $N$ est la longueur du signal considéré $x$.
\begin{figure}[!h]
\begin{center}
\includegraphics[scale = 0.4]{./../../../fig/taux_acc.png}
\caption{Courbe de charge et Taux d'accroissement}
\label{Figure1}
\end{center}
\end{figure}
\end{frame}

\begin{frame}
La figure précédente montre la présence de 3 utilisations de chauffage. On suppose qu'un chauffage a une consommation électrique d'au moins 750 W. De plus, un chauffage électrique a une courbe de charge en forme de créneaux, on va donc chercher à détecter ces créneaux. Le taux d'accroissement d'un créneau est une alternance de points positifs puis négatifs. Tous ces points sont de longueur variable.
\end{frame}

\subsection{Algorithmes}
\begin{frame}
Etape 0 :\\
Supposons qu'en $t = t_0$, la courbe de charge, $x$ dépasse 750 W. On va chercher sur les $c$ prochains points si $x(t), t_0 \le t \le t_0 + c $ devient inférieur à -750W.\\
\vspace{0.5 cm}
Etape 1:\\
\begin{itemize}
	\item S'il existe $t_1$ tel que $x(t_1) \le$ -750W alors on va chercher sur les $c$ points suivant si $x(t), t_1 \le t \le t_1 + c $ devient supérieur à 750W. On repart alors à l'étape 0 et ainsi de suite ...
	\item Sinon l'algorithme s'arrête car il n'y a plus de créneau, le chauffage s'est arrêté.
\end{itemize}
\end{frame}

\subsection{Résultats}
\begin{frame}
Nous avons également ajouté à l'algorithme de détection de chauffage par taux d'accroissement une simple détection d'appareil par franchissement de seuils fixés à 100 et 500W.
\begin{figure}[!h]
\begin{center}
\includegraphics[scale = 0.4]{./../../../fig/detection.png}
\caption{Message de détection des appareils}
\label{Figure2}
\end{center}
\end{figure}
\end{frame}

\section{Avantages et Inconvénients}

\begin{frame}{Avantages et Inconvénients}
\end{frame}

\begin{frame}
\textbf{Avantages :} \\
\begin{itemize}
	\item Algorithme très peu gourmand en complexité calculatoire.
\end{itemize}
\textbf{Inconvénients :} \\
\begin{itemize}
	\item Dépendance de seuils (Puissance et $c$).
	\item On ne peut pas détecter 2 utilisations de chauffage en même temps.
\end{itemize}
\end{frame}

%\begin{frame}
%Some text goes here
%\begin{figure}[!h]
%\begin{center}
%\includegraphics[scale = 0.4]{./../../../fig/microonde.jpg}
%\caption{Courbe de charge d'un micro-onde.}
%\label{Figure1}
%\end{center}
%\end{figure}
%\end{frame}

\end{document}
