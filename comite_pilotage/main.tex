\documentclass{beamer}

\usepackage[french]{babel}
%\usepackage[T1]{fontenc}
\usepackage[utf8]{inputenc}

\usetheme{Warsaw}
% \usecolortheme{beaver}
\setbeamertemplate{navigation symbols}{%
}
\setbeamertemplate{blocks}[rounded][shadow=false]
\setcounter{tocdepth}{2}
\title[Comité de Pilotage : Smart-EcoBox]{Comité de Pilotage :\\
Smart-EcoBox}
\author{Antoine BOUCHAIN \\
Mathias DE CACQUERAY VALMENIER \\
Melinda GOZLAN \\
Xavier LETURC \\
Marc PRATS \\
Benjamin RAMBAUD \\
Karim TOUFIQUE \\}
\institute{\textsc{ENSEIRB-MATMECA}}
%\date{14 novembre 2014}
\logo{\includegraphics[height=7mm]{../../smartee_logo.pdf}\includegraphics[height=7mm]{../../Logo_ENSEIRB.pdf}}


\begin{document}

\begin{frame}
\titlepage
\end{frame}
%--- Next Frame ---%


%\begin{frame}{Présentation du projet Technique, composition d’équipe, objectifs globaux}
%  \begin{itemize}
%    \item Créer un interface permettant à l’utilisateur de visionner ses consommations et de le corriger
%    \item ds
%  \end{itemize}
%\end{frame}
\begin{frame}{Présentation du projet Technique, composition d’équipe, objectifs globaux}
   \begin{columns}[c]
  \begin{column}{8cm}
 \begin{figure}[h!]
 \centering
 	\includegraphics[scale=0.9]{../figures/orga.pdf} 
 	\caption{Répartition des taches}
 	\label{figd3}
 \end{figure}
  \end{column}
  \begin{column}{3.7cm}
  \vspace{-50pt}
 \begin{exampleblock}{Trois équipes :} % Bloc normal
  \begin{itemize}
  \item \textbf{Recherche} classification
  \item \textbf{Développement} signal
  \item \textbf{Développement} IHM
  \end{itemize}
  \end{exampleblock}
  \end{column}
  \end{columns}
 \end{frame}
%--- Next Frame ---

\begin{frame}[c]{Situation actuelle avec planning en cours}
Gantt
\end{frame}
%--- Next Frame ---%

\begin{frame}{Le reste à faire. Stratégie pour l’atteindre}
  \begin{itemize}
    \item Mise au point du serveur Web \textsc{C, Sqlite, Php}
    \item Mise en place du client et de son interaction avec le serveur \textsc{Lighttp}
    \item Mise en service de la reconnaissance des trames \textsc{C++}
    \item Développement plus précis des algorithmes \textsc{Matlab} et implémentation en \textsc{C++}
  \end{itemize}
\end{frame}
%--- Next Frame ---%

\begin{frame}{Revue des risques et actions}
 \begin{alertblock}{Risques} % Bloc normal
  \begin{itemize}
  \item Interaction du client Web avec le serveur non finie à temps
  \item Algorithme de détection d'appareils inefficaces sous certaines conditions
  \end{itemize}
  \end{alertblock}
 \vspace{10pt}
 \begin{exampleblock}{Actions} % Bloc exampled
  \begin{itemize}
  \item Réduire l'interface à l'affichage des données
  \item Proposer une autre solution algorithmique
  \end{itemize}
  \end{exampleblock}
\end{frame}
\begin{frame}{Bilan financier (en heures)}
  \begin{itemize}
  \item Travail effectué : prévu/réalisé  
  \end{itemize}
\end{frame}
%--- Next Frame ---%
% \section{Description de la mission}
% \subsection{Banques de filtres}
% \begin{frame}{Qu'est ce qu'une banque de filtres ?}
%  \begin{columns}[c]

%  \begin{column}{7cm}
%  \begin{figure}
%  \includegraphics[scale=0.4]{../figures/fig_ana_synt_fb.pdf}
%  \label{fig_ana_synth_fb}
%  \caption{Description d'une banque de filtre}
%  \end{figure}
%  \end{column}

%  \begin{column}{5cm}
%  \begin{block}{Description}
%  \begin{itemize}
%  \item AFB : Analysis Filter Bank
%  \item SFB : Synthesis Filter Bank
%  \item $X(z)$ : signal d'entrée
%  \item $\tilde{X}(z)$ : signal de sortie
%  \item $M$ : nombre de canaux ou utilisateurs
%  \end{itemize}
%  \end{block}
%  \end{column}
%  \end{columns}
%  \end{frame}
 
 
%  % NEW FRAME%
%  \begin{frame}{Traitement obtenu grâce à une banque de filtres} 
%   \begin{columns}[c]
%    \begin{column}{5cm}
%    \begin{figure}
% 	\label{figd2a}
%   	\includegraphics[width=150pt]{../figures/ana_synth_fb.pdf}
%  	\label{fig_ana_synth_fb}
%  	\caption{sur le signal global}
%  \end{figure}
%  \end{column}
%  \begin{column}{5cm}
%    \begin{figure}
%  \label{figd2b}
% \includegraphics[width=150pt]{../figures/ana_channel_fb.pdf}
% \caption{sur les signaux de chaque canal}
%    \end{figure}
%  \end{column}
%  \end{columns}
%  \end{frame}
 
 
 
%  \begin{frame}{Type de banques de filtres}
%  \textbf{ Types de banques de filtres}
%   \begin{columns}[c]
%  \begin{column}{5cm}
%  \begin{block}{Par résultats obtenus} % Bloc normal
%  \begin{itemize}
%  \item \textbf{Uniforme} : chaque utilisateur a la même bande passante
%  \item \textbf{Non Uniforme} : chaque utilisateur peut avoir une largeur de bande différente
%  \end{itemize}
%  \end{block}
%  \end{column}
%  \begin{column}{5cm}
% \begin{block}{Par type de filtre passe bas utilisé} % Bloc normal
%  \begin{itemize}
%  \item \textbf{Filtre FIR} : utilisation d'un filtre à réponse impulsionnelle finie
%  \item \textbf{Filtre IIR} : utilisation d'un filtre à réponse impulsionnelle infinie
%  \end{itemize}
%  \end{block}
%  \end{column}
%  \end{columns}
%  \end{frame}
 
 
 
% \subsection{Banques de filtres uniformes}
% \begin{frame}{Approche basique}
%   \begin{columns}[c]
%  \begin{column}{5cm}
%  \vspace{-20pt}
% \begin{figure}[h!]
% \centering
% 	\includegraphics[scale=0.55]{../figures/fig_afb_uni_basic.pdf}
% 	\caption{Représentation d'une banque de filtres basique}
% 	\label{figd3}
% \end{figure}
%  \end{column}
%  \begin{column}{5cm}
% \begin{exampleblock}{Avantage :} % Bloc normal
%  \begin{itemize}
%  \item Simple à réaliser
%  \end{itemize}
%  \end{exampleblock}
% \vspace{10pt}
% \begin{alertblock}{Inconvénient :} % Bloc normal
%  \begin{itemize}
%  \item Importante complexité calculatoire
%  \end{itemize}
%  \end{alertblock}
%  \end{column}
%  \end{columns}
% \end{frame}




% \subsubsection{Méthode polyphase}
% \begin{frame}{Approche polyphase}
%   \begin{columns}[c]
%  \begin{column}{5cm}
% \begin{figure}[h!]
% \centering
% 	\includegraphics[scale=0.55]{../figures/fig_afb_poly_decimator.pdf} 
% 	\caption{Représentation d'une banque de filtre basique}
% 	\label{figd3}
% \end{figure}
%  \end{column}
%  \begin{column}{5cm}
%  \vspace{-50pt}
% \begin{exampleblock}{Avantages :} % Bloc normal
%  \begin{itemize}
%  \item Faible complexité calculatoire
%  \item Tire avantage de l'algorithme \textsc{Fft} : peu de multiplieurs requis
%  \end{itemize}
%  \end{exampleblock}
% \vspace{10pt}
% \begin{alertblock}{Inconvénient :} % Bloc normal
%  \begin{itemize}
%  \item Plus compliqué que l'approche basique
%  \end{itemize}
%  \end{alertblock}
%  \end{column}
%  \end{columns}
% \end{frame}
% \subsection{Banques de filtres uniformes IIR}
% \begin{frame}{Structure IIR polyphase}
%   \begin{columns}[c]
%    \begin{column}{5cm}
%    \begin{figure}
% 	\label{figd2a}
%   	\includegraphics[width=150pt]{../figures/fig_filter_iir_poly.pdf}
%  	\label{fig_ana_synth_fb}
%  	\caption{créée grâce au logiciel du Pr Harris}
%  \end{figure}
%  \end{column}

%  \begin{column}{5cm}
%    \begin{figure}
%  \label{figd2b}
% \includegraphics[width=150pt]{../figures/fig_filter_poly.pdf}
% \caption{utilisée dans les banques de filtres.}
%    \end{figure}
%  \end{column}
%  \end{columns}
% \end{frame}



% \section{Résultats de la mission}
% \subsection{Avantages des structures IIR polyphases}
% \subsubsection{Nombre de coefficients}
% \begin{frame}[t]{Nombre de coefficients}
% \vspace{-30pt}
%   \begin{columns}[c]
%    \begin{column}{5cm}
%    \begin{figure}
% 	\label{figd2a}
%   	\includegraphics[width=170pt]{../figures/fig_freq_phase_100.pdf}
%  	\label{fig_ana_synth_fb}
%  	\caption{Réponse fréquentielle et en phase pour un filtre IIR équivalent à un filtre FIR de 100 coefficients}
%  \end{figure}
%  \end{column}

%  \begin{column}{5cm}
%    \begin{figure}
%  \label{figd2b}
% \includegraphics[width=150pt]{../figures/fig_carac_IIR_attenuation.pdf}
% \caption{Effets de la variation du nombre de coefficients et de l'importance de l'atténuation hors bande}
%    \end{figure}
%  \end{column}
%  \end{columns}
% \end{frame}
% \subsubsection{Gain en complexité d’une structure IIR sur une structure FIR}
% \begin{frame}{Gain en complexité}
%   \begin{columns}[c]
%    \begin{column}{6cm}
%    \begin{figure}
% 	\label{figd2a}
%   	\includegraphics[width=180pt]{../figures/fig_complex_time.pdf}
%  	\label{fig_ana_synth_fb}
%  	\caption{Comparaison de complexité entre banques de filtres FIR et IIR}
%  \end{figure}
%  \end{column}

%  \begin{column}{5cm}
%   \begin{block}{Avantages}
%   \begin{itemize}
%   \item Moins de coefficients par branche
%   \item Complexité moindre
%   \item Moins d'additionneurs et multiplieurs requis
%   \end{itemize}
%   \end{block}
%  \end{column}
%  \end{columns}
% \end{frame}
% \subsection{Performances des banques de filtres IIR par rapport aux banques de filtres FIR}
% \subsubsection{Modulation QPSK}
% \begin{frame}[t]{Nombre de coefficients}
% Taux d'erreurs binaires en fonction du rapport signal à bruit : théorique; pratique FIR polyphase; IIR polyphase pour un temps symbole de :
% \vspace{-30pt}
%   \begin{columns}[c]
%    \begin{column}{5cm}
%    \begin{figure}
% 	\label{figd2a}
%   	\includegraphics[width=150pt]{../figures/fig_ber_comp_1Ts.pdf}
%  	\label{fig_ana_synth_fb}
%  	\caption{$T_s=\frac{1}{F_s}$}
%  \end{figure}
%  \end{column}

%  \begin{column}{5cm}
%    \begin{figure}
%  \label{figd2b}
% \includegraphics[width=170pt]{../figures/fig_ber_comp_2Ts.pdf}
% \caption{$T_s=\frac{2}{F_s}$}
%    \end{figure}
%  \end{column}
%  \end{columns}
% \end{frame}
% \subsubsection{Effet de la bande de garde}
% \begin{frame}[t]{Effet de la bande de garde}
% \vspace{-30pt}
%   \begin{columns}[c]
%    \begin{column}{5cm}
%    \begin{figure}
% 	\label{figd2a}
%   	\includegraphics[width=150pt]{../figures/fig_3D_BER_Ts.pdf}
%  	\label{fig_ana_synth_fb}
%  	\caption{Taux d'erreur binaire en fonction du temps symbole et du SNR pour une banque de filtres FIR et une banque de filtres IIR}
%  \end{figure}
%  \end{column}

%  \begin{column}{5cm}
%    \begin{figure}
%  \label{figd2b}
% \includegraphics[width=170pt]{../figures/fig_3D_BER_Ts_comp.pdf}
% \caption{Comparaison des résultats avec les valeurs théoriques du BER}
%    \end{figure}
%  \end{column}
%  \end{columns}
% \end{frame}
% \subsubsection{Effet de la largeur des pics}
% \begin{frame}[t]{Effet de la largeur des pics}
% \vspace{-30pt}
%   \begin{columns}[c]
%    \begin{column}{5cm}
%    \begin{figure}
% 	\label{figd2a}
%   	\includegraphics[width=150pt]{../figures/fig_comparison_ber_nbcoefs_40dB.pdf}
%  	\label{fig_ana_synth_fb}
%  	\caption{BER versus nombre de coefficients par branche d'un filtre IIR polyphase avec atténuation hors bande de $40dB$}
%  \end{figure}
%  \end{column}

%  \begin{column}{5cm}
%    \begin{figure}
%  \label{figd2b}
% \includegraphics[width=170pt]{../figures/fig_ori_filters_synth_FIR_IIR.pdf}
% \caption{avant banque de filtres d'analyse, après banque de filtres de synthèse}
%    \end{figure}
%  \end{column}
%  \end{columns}
% \end{frame}









% \begin{frame}[c]{Conclusion}
% \begin{itemize}
% \item Réduction des performances par les filtres IIR polyphase
% \item Choix du filtre primordial
% \item[]
% \item Découverte du monde de la recherche
% \item Amélioration de l'anglais
% \item Découverte d'une capitale européenne
% \end{itemize}
% \end{frame}
%--- Next Frame ---%






\end{document}
