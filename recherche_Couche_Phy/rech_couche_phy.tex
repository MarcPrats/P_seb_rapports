\documentclass[10pt,a4paper]{article}
\usepackage[utf8]{inputenc}
\usepackage[T1]{fontenc}
\usepackage[french]{babel}
\usepackage{enumitem}
%\frenchbsetup{StandardLists=true}
\usepackage{caption}
\usepackage{graphicx}
\usepackage{amsmath}
\usepackage{subfigure}
\usepackage{appendix}
\usepackage{listings}
\usepackage{multicol}
\usepackage{moreverb}
\usepackage{epsfig}
\usepackage{textcomp}
\usepackage{hyperref}
\usepackage{xcolor}
      %%%%%%%%%%%%%%%%  inclure la source %%%%%%%%%%%%%%%%%%%%
  \newcommand*\styleC{\fontsize{9}{10pt}\usefont{T1}{cmtt}{m}{n}\selectfont }
  \newcommand*\styleD{\fontsize{9}{10pt}\usefont{T1}{cmtt}{m}{n}\selectfont }

      \makeatletter
      % on fixe le langage utilisé
      \lstset{language=matlab, breaklines=true}
      \edef\Motscle{emph={\lst@keywords}}
      \expandafter\lstset\expandafter{%
       \Motscle}
      \makeatother


      \definecolor{Vvert}{rgb}{0.133,0.545,0.133}

      \lstset{emphstyle=\color{blue}, % les mots réservés de matlab en bleu
      basicstyle=\styleC,
      keywordstyle=\cmtt,
      commentstyle=\color{Vvert}\styleD, % commentaire en gris
      numberstyle=\tiny\color{red},
      numbers=left,
      numbersep=10pt,
      lineskip=0.7pt,
      showstringspaces=false}
      %  % inclure le fichier source
      \newcommand{\FSource}[1]{%
      \lstinputlisting[texcl=true]{#1}
      }

\setlength{\textwidth}{16cm} % Largeur du texte
\setlength{\textheight}{24cm} % Hauteur du texte
\setlength{\evensidemargin}{-0.2cm} % Taille des marges pour les pages paires
\setlength{\oddsidemargin}{-0.2cm} % Taille des marges pour les pages impaires
\setlength{\topmargin}{-1.5cm} % Taille de l'ent\^ete

\title{PR307 Smart-EcoBox : \\
Recherche Couche Physique}
\author{
DE CACQUERAY VALMENIER Mathias <mdecacquerayvalmenie@enseirb-matmeca.fr> \\
GOZLAN Melinda <mgozlan@enseirb-matmeca.fr>\\}


\begin{document}
\maketitle

\begin{figure}[ht]
\begin{center}
\noindent \includegraphics[scale = 0.25]{../../Logo_ENSEIRB.pdf}
\end{center}
\end{figure}
\begin{figure}[ht]
\begin{center}
\noindent \includegraphics[scale = 0.25]{../../smartee_logo.pdf}
\end{center}
\end{figure}


\newpage

\tableofcontents

\newpage
\begin{abstract}
Ce rapport à pour objectif de proposer la meilleur solution pour dialoguer entre capteurs et BOX en sans fils.
\end{abstract}

\section{Zi-Wave}
Technologie permettant la communication entre les périphériques et un serveur. Très complexe. Voici un résumé :
\begin{itemize}
\item Modulation du signal de type 2-FSK
\item Réseau maillé dont la gestion des capteurs se fait par un contrôleur
\item Réseau sécurisé car le contrôleur attribut un identifiant unique à chaque capteur
\item Possibilité de joindre un élément éloigné spatialement grâce à la structure en maillage
\item Protocole radio bidirectionnel
\item Interopérabilité du système domotique
\end{itemize}
\textbf{Inconvénients :}
\begin{itemize}
\item La communication bidirectionnelle n’est pas nécessaire dans le cadre des objectifs qui ont été fixé; seule la partie transmission nous est utile. 
\end{itemize}
\section{HomeEasy}
HomeEasy est le protocole utilisé par Chacon, DI-O.
\begin{itemize}
\item Communication uni-directionnelle
\item Bande 868 MHz
\item Facile à mettre en œuvre.
\end{itemize}

\section{Proposition de capteur HomeEasy}
\subsection{Température}
N'importe qu'elle capteur DIO fait l'affaire. Voici un exemple :\\
Capteur TX35IT+.\\
\url{http://www.conrad.fr/ce/fr/product/094466/Capteur-thermomtre-La-Crosse-Technology-TX35IT?queryFromSuggest=true}
\subsection{Débit D'eau}
Nous avons trouvé cette solution, qui à l'avantage d'être simple à mettre en oeuvre et d'être peut couteuse. Elle convient pour l'utilisation requise.\\
\url{http://www.poulpy.com/2011/11/releve-de-consommation-deau-low-cost-en-rf/}
\end{document}


%\begin{figure}[!h]
%\begin{center}
%\includegraphics[scale = 0.4]{./fig/fig1.jpg}
%\caption{figure}
%\label{Figure}
%\end{center}
%\end{figure}
