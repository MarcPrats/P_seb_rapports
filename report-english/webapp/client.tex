%%%%%%%%%%%%%%%%%%%%%%%%%%%%%%%%%%%%%%%%%%%%%%%%%
%       Client                                  %
%                                               %
%%%%%%%%%%%%%%%%%%%%%%%%%%%%%%%%%%%%%%%%%%%%%%%%%
\subsection{Client} % Interface Client

This subsection introduce the design of the web interface, especially the structure and implementation of the main features.
% expliquer l'utilité de l'interface

%%%%%%%%%%%%%%%%%%%%%%%%%%%%%%%%%%%%%%%%%%%%%%%%%%%%%%%%%%%%%%%%%%%%%%%%%%%%%
\subsubsection{Structure}
    % serveur léger -> traitement côté client -> JS -> Angular
    % parler MVC avec Angular
    % n3-chart : n3-chart makes creating beautiful charts for AngularJS application easy and semantic. It is built on top of D3.js library.
    %
    % schéma du site
    Since the server has to be light and only provide the data, this interface has been built with the framework AngularJS 1.3. Indeed, it enables efficient and easily maitainable client-side data processing. In addition, it is based on the Model-View-Controller architectural pattern (MVC) that allows easy maintenance of the code \textbf{(ajouter d'autres points positif)}. The curves has been built with n3-chart. It makes creating beautiful charts for AngularJS application easy and semantic and it is built on top of D3.js library.
    
    The interface has four main pages reachable via the navigation bar: Homepage, Power Consumption page, Temperature page, Configuration page. Let us add the login page to access the web app and the page to set up and configure the box.
     
    \begin{figure}[!h]
        \centering
        % pour changer la couleur
        %%  cf main.tex à \tikzstyle pour tout le document
        %   ou
        %%  cf webapp.tex pour savoir comment changer la couleur d'un node
        \begin{tikzpicture}[node distance=1.5cm]
    
        \node (startConf) [startstop] {Setting up and Configuring};
        \node (login) [process, below of=startConf] {Log in};
        \node (home) [process, below of=login] {Homepage};
        \node (conf) [process, below of=home] {Configuration};
        \node (conso) [process, right of=conf, xshift=2cm] {Power Consumption};
        \node (temp) [process, left of=conf, xshift=-2cm] {Temperature};

        \draw [arrow] (startConf) -- (login);
        \draw [arrow] (login) -- (home);
        \draw [arrow] (home) -- (conf);
        \draw [arrow] (home) -- (conso);
        \draw [arrow] (home) -- (temp);
        \end{tikzpicture}
    \end{figure}

     
%%%%%%%%%%%%%%%%%%%%%%%%%%%%%%%%%%%%%%%%%%%%%%%%%%%%%%%%%%%%%%%%%%%%%%%%%%%%%
\subsubsection{Homepage}

The homepage provides quick access to the general condition of the system.
The main information relating to the operation of the system and an overview of the data sent by the sensors is found on this page.
The homepage includes four main blocks:%, - one for electric consumption and one for temperature, one for notifications and one for the states of registered devices - where the most important data will be indicated.
    \paragraph{Power Consumption} 
    This block provides acces to two main information: the instantaneous and accumulated power consumption of the house. An icon related to the theme of meteorology indicates the evolution of these data. % TODO Button to switch and see the cost
    \paragraph{Temperature}
    This block provides acces to two main information: the outside and inside temperature. 
    \paragraph{Notifications}
    This block allows the user to monitor changes of sensors status and registered type of devices status. The system notifies the user when a status is changed.
    %State sensor
    %State registered devices
    \paragraph{List of Devices} % Liste des types d’appareils enregistrés
    This block provide access of all registered type of devices and their status.
    %know the state of all registered type of devices.

%%%%%%%%%%%%%%%%%%%%%%%%%%%%%%%%%%%%%%%%%%%%%%%%%%%%%%%%%%%%%%%%%%%%%%%%%%%%%
\subsubsection{Power Consumption}

    \paragraph{Curve Page}
    % suivre evolution de sa conso
    %   courbe de conso
    %   courbe de coût associé
    %   coût cumulé sur la période.
    The user can access to the evolution of his domestic power consumption thanks to three information: the power consumption curve, the matched cost curve and the accumulated cost.
    
    \paragraph{Estimated Electricity Bill}
    An estimated electricity bill can be viewed on this page. It is inspired by a real electricity bill and allows the user to know an estimate depending on his behaviour.
%%%%%%%%%%%%%%%%%%%%%%%%%%%%%%%%%%%%%%%%%%%%%%%%%%%%%%%%%%%%%%%%%%%%%%%%%%%%%
\subsubsection{Temperature}
    % suivre evolution de la temperature au sein sa maison
    %   courbes de temperature
    %   plusieurs capteurs
    For each temperature sensors, a temperature curve can be displayed allowing the user to access to the evolution of the temperature within the house. 
%%%%%%%%%%%%%%%%%%%%%%%%%%%%%%%%%%%%%%%%%%%%%%%%%%%%%%%%%%%%%%%%%%%%%%%%%%%%%
\subsubsection{Configuration Pages}
    
    \paragraph{Setting up and Configuring} % configuration à l'allumage
    
    When the user turn on his box for the first time, he need to register by providing his data: name, email, password, city, and his EDF subscription. Then, the system automatically displays the different sensors detected thanks to the signal processing algorithm. The user have to set up the system by providing the name, location and type of each sensors. % (a color-based recognition could be used)
    
    \paragraph{Current Usage} % pendant l'utilisation
    
    \subparagraph{"My Account" Page:} 
    The user can access to his personal information and change them. This includes his name, password, email, city and his EDF subscription. Similarly, the profile appears on the homepage.
    
    \subparagraph{"My Sensors" Page:}
    This page is used to list the different sensors. Each sensor can be renamed by the user, may also indicate a location specified by the user (eg children's room) and specifies a type of sensor that is recognized by a color code. The SNR (Signal-to-Noise Ratio) is also shown in the form of bars to show the state of a sensor (like mobile phone signal bars). He can edit the sensor information using the "Edit" button.