%%%%%%%%%%%%%%%%%%%%%%%%%%%%%%%%%%%%%%%%%%%%%%%%%%%%%%%%%%%%%%%%%%%%%%%%%%%%%%%%%%%%%%%
%
%   Research                   
%
%%
%%%%%%%%%%%%%%%%%%%%%%%%%%%%%%%%%%%%%%%%%%%%%%%%%%%%%%%%%%%%%%%%%%%%%%%%%%%%%%%%%%%%%
\section{Research}

The strength of our device lies in its ability to determine precisely which device is turned on or not. Thankfully, each device has a specific load curve, and analyzing different points of the load curve allows us to determine when a device turns on. Two different approach were proceeded, a macroscopic and a microscopic one. The main difference between them are the sampling frequency and the focused points of the curves. The macroscopic approach require about 1Hz and study the whole form of the load curve whereas, for the microscopic one, it require between 5 and 10 kHz and study the transient response.

%%%%%%%%%%%%%%%%%%%%%%%%%%%%%%%%%%%%%%%%%%%%%%%%%
%       Macroscopic analysis of load curve      %
%%%%%%%%%%%%%%%%%%%%%%%%%%%%%%%%%%%%%%%%%%%%%%%%%
%%%%%%%%%%%%%%%%%%%%%%%%%%%%%%%%%%%%%%%%%%%%%%%%%
%       Macroscopic analysis of load curve      %
%   Antoine                                     %
%%%%%%%%%%%%%%%%%%%%%%%%%%%%%%%%%%%%%%%%%%%%%%%%%
\subsection{Macroscopic analysis of a load curve}
A fast algorithm is needed and that can be called at any moment to detect a device. It is based on a study of the load curve. In case the results of this approach are doubtful, the microscopic algorithms are used.

The macroscopic algorithms are based on the differentiate function, which helps detect a jump in the load curve. Indeed the difference between two points of a jump is directly the consumption of the device, as shown in \ref{fig1} where the grouped square are a heater use.


\begin{figure}[h]
 \begin{center}
   \subfloat[Example of a real load curve]{
   \includegraphics[width=0.45\textwidth]{figures/fig_antoine_5.png}
   \label{fig1}
   }
   \subfloat[Estimation of a heater consumption]{
   \includegraphics[width=0.45\textwidth]{figures/fig_antoine_6.png}
   \label{fig2}
   }
   \caption{Macroscopic Analysis}
   \label{fig-1-1-2}
 \end{center}
\end{figure}

Several steps of consumption were defined: 100, 500 and 750W which mean small consumption device (such as lights), medium consumption (oven) and heavy consumption (heater). The load curve reaching one of these values means the device is turning on, and reaching the same negative value means the device is turning off. With this basic approach some devices can be defined, but it is not possible to differentiate an unknown device consuming 500W from an oven for example.

Therefore, a second algorithm was developed that analyzes the signature load curve of each device. Every devices have a specific and unique load curve when they turn on. If a data base were created with these signatures, they could be used to determine which device has just turn on. Spotting when the devices turn off is simple as it corresponds to negative values of the differentiate function.

An example will help make the description clearer. A heater power consumption load curve is basically a square repetition. This shape is due to the periodicity of the device. By detecting such a wave form in the load curve, then we can detect heater consumption and estimate the consumption of this device. These results are shown on the figure \ref{fig2}.


Non periodical devices are processed in the same way, such as a dishwasher or a washing machine. But at the moment, only the dishwasher has been implemented.




%%%%%%%%%%%%%%%%%%%%%%%%%%%%%%%%%%%%%%%%%%%%%%%%%
%       Microscopic analysis of load curve      %
%%%%%%%%%%%%%%%%%%%%%%%%%%%%%%%%%%%%%%%%%%%%%%%%%
%%%%%%%%%%%%%%%%%%%%%%%%%%%%%%%%%%%%%%%%%%%%%%%%%
%       Microscopic analysis of load curve      %
%   Xavier                                      %
%%%%%%%%%%%%%%%%%%%%%%%%%%%%%%%%%%%%%%%%%%%%%%%%%

\subsection{Microscopic analysis of the load curve}
%Xavier
The sampling frequency used in the case of a microscopic analysis of the load curve is between 5 and 10k$Hz$. Such a frequency is required to access the transient responses of the devices. Using such a high sampling frequency is a way to avoid a difficulty encountered in macroscopic analysis coming from the fact that an appliance is generally switched on while other appliances are already in use. 
\\

An important point is that the transient response is linked with the physical role of the device. Studies have then been conducted to isolate characteristic parameters of the appliance from the transient response. Here is an example of a transient response :
\\
% Figure

\begin{figure}[H]
\centering
\includegraphics[scale=0.5]{figures/transient_response.png}
\caption{Transient response}
\label{fig:boxDescription}
\end{figure}



[2] address the problem of classification of transient signals which are modelled as a Smooth transition autoregressive model. Figure x shows some abrupt transitions on the transient signal, which are called breakpoints. Let $K$ denote the number of breakpoints, and $\underline{\tau} = [\tau_1,...,\tau_K]$ denote the sequence of the $K$ breakpoints. By convention, $[\tau_0;\tau_{K+1}]$ represents the observation interval of the signal. 
\\

 Therefore, the signal is modelled as a linear combination of regression functions $\{m_k\}_{k\in{1,K+1}}$.~$m_k$ is the regression model of the signal over the segment $[\tau_{k-1};\tau_{k}]$. Then, the signal is expressed as

\begin{equation}
\forall t\in [\tau_0;\tau_{K+1}],m(t)= \sum_{k=1}^{K+1}p_k(t)m_k(t)
\end{equation}

where $p_k(t)$ is the weight function of the $k^{th}$ regression model which can be expressed as follow
\begin{equation}
\forall k\in [1,K+1], p_k(t)= \pi_{\eta_{k-1}}(t-\tau_{k-1})-\pi_{\eta_{k}}(t-\tau_{k})
\end{equation}

$\pi_{\eta_{k}}$ is the transition function of the $k^{th}$ breakpoint, which depends on a vector of parameter $\eta_{k}$. In the model described in [1], $\pi_{\eta_{k}}$ is a sigmoide function, whose expression is 
\begin{equation}
\pi_{\eta_{k}} = \frac{1}{1+exp(\frac{t}{\lambda_k})}
\end{equation}

The only element in $\eta_{k}$ is then $\lambda_k$, which is a scale parameter. Let $\underline{\lambda}$ denote the concatenation of all the scale parameters $\underline{\lambda} = [\lambda_1,...,\lambda_K]$.
\\

 Regression model are considered in [2] as polynomial 

\begin{equation}
m_k(t)= \sum_{q=0}^{Q_k} \beta^{(q)} _kt^q
\end{equation}

$\beta^{(q)} _k$ is the $q^{th}$ order regression coefficient of the $k^{th}$ regression model, and $Q_k$ is the order of $m_k(t)$. As it was the case for $\lambda$, $\beta$ regroups all the $\beta_k$ coefficients and $Q$ all the $Q_k$ coefficients.
\\

A common model for the observed signal is 


\begin{equation}
s(t)=m(t) + \epsilon (t)
\end{equation}

$\epsilon (t)$ is an additive white Gaussian noise whose variance is $\sigma^2$. The different parameters which have to be evaluated are then :
\begin{equation}
\theta=\{\underline{\lambda},\underline{\tau},\underline{\beta},\underline{Q},\sigma ^2,K\}
\end{equation}
\\


The approach presented in [2] is a Bayesian one. The $posterior$ probability density function of $\theta$ is obtained thanks to the Bayes formula. The algorithm used to sample the distibution is a Reversible Jump Markov chain Monte Carlo (RJMCMC). The algorithm consists in a main loop which is repeated a high number of time. The higher the number of iterations, the more accurate the algorithm. A new set of values is purposed for $\theta$ during each iteration. The different kind of movements that can be purposed concerning $\theta$ are :
\begin{itemize}
\item Birth of a transition
\item Death of a transition
\item Division of an existing transition
\item Fusion of two existings transitions
\item Update of a transition
\end{itemize}
The movement is then accepted or rejected with a certain probability, which depends on Hasting-Green-Metropolis ratio. 
\\

The classification of electrical devices is then done thanks to $\theta$. Since that study was more a theoretical than a practical work, it has not been implemented yet on the box, but is functionnal for very basics signals.


