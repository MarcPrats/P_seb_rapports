\documentclass[10pt,a4paper]{article}
\usepackage[utf8]{inputenc}
\usepackage[T1]{fontenc}
\usepackage[french]{babel}
\usepackage{enumitem}
%\frenchbsetup{StandardLists=true}
\usepackage{caption}
\usepackage{graphicx}
\usepackage{amsmath}
\usepackage{subfigure}
\usepackage{appendix}
\usepackage{listings}
\usepackage{multicol}
\usepackage{moreverb}
\usepackage{epsfig}
\usepackage{textcomp}


    \usepackage{xcolor}
      %%%%%%%%%%%%%%%%  inclure la source %%%%%%%%%%%%%%%%%%%%
  \newcommand*\styleC{\fontsize{9}{10pt}\usefont{T1}{cmtt}{m}{n}\selectfont }
  \newcommand*\styleD{\fontsize{9}{10pt}\usefont{T1}{cmtt}{m}{n}\selectfont }

      \makeatletter
      % on fixe le langage utilisé
      \lstset{language=matlab, breaklines=true}
      \edef\Motscle{emph={\lst@keywords}}
      \expandafter\lstset\expandafter{%
       \Motscle}
      \makeatother


      \definecolor{Vvert}{rgb}{0.133,0.545,0.133}

      \lstset{emphstyle=\color{blue}, % les mots réservés de matlab en bleu
      basicstyle=\styleC,
      keywordstyle=\cmtt,
      commentstyle=\color{Vvert}\styleD, % commentaire en gris
      numberstyle=\tiny\color{red},
      numbers=left,
      numbersep=10pt,
      lineskip=0.7pt,
      showstringspaces=false}
      %  % inclure le fichier source
      \newcommand{\FSource}[1]{%
      \lstinputlisting[texcl=true]{#1}
      }

\setlength{\textwidth}{16cm} % Largeur du texte
\setlength{\textheight}{24cm} % Hauteur du texte
\setlength{\evensidemargin}{-0.2cm} % Taille des marges pour les pages paires
\setlength{\oddsidemargin}{-0.2cm} % Taille des marges pour les pages impaires
\setlength{\topmargin}{-1.5cm} % Taille de l'ent\^ete

\title{PR307 Smart-EcoBox : \\
Spécifications pour la partie signal}
\author{
BOUCHAIN Antoine <abouchain@enseirb-matmeca.fr> \\
DE CACQUERAY VALMENIER Mathias <mdecacquerayvalmenie@enseirb-matmeca.fr> \\
GOZLAN Melinda <mgozlan@enseirb-matmeca.fr>\\
LETURC Xavier <xleturc@enseirb-matmeca.fr> \\
PRATS ESCRIBANO Marc <mpratsescribano@enseirb-matmeca.fr> \\
RAMBAUD Benjamin <brambaud@enseirb-matmeca.fr> \\
TOUFIQUE Karim <ktoufique@enseirb-matmeca.fr> \\ }


\begin{document}
\maketitle

\begin{figure}[ht]
\begin{center}
\noindent \includegraphics[scale = 0.25]{../../Logo_ENSEIRB.pdf}
\end{center}
\end{figure}
\begin{figure}[ht]
\begin{center}
\noindent \includegraphics[scale = 0.25]{../../smartee_logo.pdf}
\end{center}
\end{figure}


\newpage

\tableofcontents

\newpage

\section{Récupération des données}

Nous récupérons les données en provenance des capteurs sous forme de trames via trois drivers possibles:
\begin{itemize}
\item UHD utilisé par Ettus Research USRP 
\item RTL présent dans certains tuners TNT USB
\item TEST  pour relancer des samples déjà enregistrés\\
\end{itemize}

Les fabricants fournissent des codes permettent d'enregistrer des samples dans des fichiers.
On peut ainsi récupérer dans un buffer des samples.
Nous choisissons de traiter 4096 par 4096 samples qui seront ensuite traités dans la partie "création de burst" de nos codes.\\

L'avantage de ce mode de fonctionnement permet d'introduire un quelconque autre driver sans à avoir à modifier le code implémenté pour la création de burst et l'interprétation également.

\section{Détection de burst}

Pour pouvoir détecter le burst parmi les 4096 samples pour la suite de nos traitements, nous calculons l'énergie sur la durée totale du buffer puis nous la comparons à un seuil égal à 3*$\sigma$ avec $\sigma$ représentant la variance du bruit.\\ Pour fixer la valeur de $\sigma$ nous calculons la moyenne de l'énergie sur 50 buffers successifs.\\ 
Si l'énergie d'un buffer est supérieure à ce seuil, nous gardons ce buffer contenant de l'information sinon nous le rejetons et nous réitérons l'opération.\\

Pour traiter le cas où une partie du burst est présente au début ou à la fin du buffer. Nous concaténons toujours le buffer suivant et précédent  au buffer consécutif où de l'information a été détectée.
Le burst est donc composé de bruit, de signal d'information + bruit et de bruit .


 
\end{document}


%\begin{figure}[!h]
%\begin{center}
%\includegraphics[scale = 0.4]{./fig/fig1.jpg}
%\caption{figure}
%\label{Figure}
%\end{center}
%\end{figure}
