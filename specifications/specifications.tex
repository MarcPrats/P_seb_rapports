\documentclass[10pt,a4paper]{article}
\usepackage[utf8]{inputenc}
\usepackage[T1]{fontenc}
\usepackage[french]{babel}
\usepackage{enumitem}
%\frenchbsetup{StandardLists=true}
\usepackage{caption}
\usepackage{graphicx}
\usepackage{amsmath}
\usepackage{subfigure}
\usepackage{appendix}
\usepackage{listings}
\usepackage{multicol}
\usepackage{moreverb}
\usepackage{epsfig}
\usepackage{textcomp}


    \usepackage{xcolor}
      %%%%%%%%%%%%%%%%  inclure la source %%%%%%%%%%%%%%%%%%%%
  \newcommand*\styleC{\fontsize{9}{10pt}\usefont{T1}{cmtt}{m}{n}\selectfont }
  \newcommand*\styleD{\fontsize{9}{10pt}\usefont{T1}{cmtt}{m}{n}\selectfont }

      \makeatletter
      % on fixe le langage utilisé
      \lstset{language=matlab, breaklines=true}
      \edef\Motscle{emph={\lst@keywords}}
      \expandafter\lstset\expandafter{%
       \Motscle}
      \makeatother


      \definecolor{Vvert}{rgb}{0.133,0.545,0.133}

      \lstset{emphstyle=\color{blue}, % les mots réservés de matlab en bleu
      basicstyle=\styleC,
      keywordstyle=\cmtt,
      commentstyle=\color{Vvert}\styleD, % commentaire en gris
      numberstyle=\tiny\color{red},
      numbers=left,
      numbersep=10pt,
      lineskip=0.7pt,
      showstringspaces=false}
      %  % inclure le fichier source
      \newcommand{\FSource}[1]{%
      \lstinputlisting[texcl=true]{#1}
      }

\setlength{\textwidth}{16cm} % Largeur du texte
\setlength{\textheight}{24cm} % Hauteur du texte
\setlength{\evensidemargin}{-0.2cm} % Taille des marges pour les pages paires
\setlength{\oddsidemargin}{-0.2cm} % Taille des marges pour les pages impaires
\setlength{\topmargin}{-1.5cm} % Taille de l'ent\^ete

\title{PR307 Smart-EcoBox : \\
Spécifications}
\author{LETURC Xavier <xleturc@enseirb-matmeca.fr> \\
DE CACQUERAY VALMENIER Mathias <mdecacquerayvalmenie@enseirb-matmeca.fr> \\
BOUCHAIN Antoine <abouchain@enseirb-matmeca.fr> \\
RAMBAUD Benjamin <brambaud@enseirb-matmeca.fr> \\
PRATS ESCRIBANO Marc <mpratsescribano@enseirb-matmeca.fr> \\
TOUFIQUE Karim <ktoufique@enseirb-matmeca.fr> \\
GOZLAN Melinda <mgozlan@enseirb-matmeca.fr> }


\begin{document}

\maketitle

\begin{figure}[ht]
\begin{center}
%\noindent \includegraphics[scale = 0.3]{../../../Logo_ENSEIRB.pdf}
\end{center}
\end{figure}


\newpage

\tableofcontents

\newpage

\section{Organisation}
Mathias de Cacqueray-Valmenier est chef de projet, il est secondé pour la partie informatique par Marc Prats Escribano.
\subsection{Partie Informatique}
La partie informatique de ce projet comporte 3 axes :\\
\textbf{Axe 1 :} Trouver une solution technique concernant le stockage des données récupérées depuis les capteurs. Ces
données peuvent être volumineuses.
\begin{itemize}
\item JP
\end{itemize}
\textbf{Axe 2 :} Développer des vues qui permettent de visualiser les données stockées de façon intelligible.
\begin{itemize}
\item JP
\end{itemize}
\textbf{Axe 3 :} Permettre aux utilisateurs de configurer l’outil : changer les noms des appareils éléctriques, configurer la
façon dont les données sont affichées, ...
\begin{itemize}
\item JP
\end{itemize}
\subsection{Partie Signal}
La partie signal/communications numériques de ce projet comporte 3 axes :\\
\textbf{Axe 1 :} Etude bibliographique et implémentations matlab des méthodes/algorithmes permettant l’identification
d’appareils électriques via l’analyse des signaux qu’ils génèrent sur le réseau électrique lors de leur allumage,
\begin{itemize}
\item Antoine Bouchain
\item Xavier Leturc
\end{itemize}
\textbf{Axe 2 :} Propositions de couches physiques (box + capteurs) fiable, efficace et cohérente avec des communications
dans la bande ISM 433MHz. La faisabilité de ces propostions devra être montrée par simulation Matlab.
\begin{itemize}
\item Mélinda Gozlan
\item Mathias de Cacqueray-Valmenier
\end{itemize}
\textbf{Axe 3 :} Réalisation d’un démonstrateur avec des capteurs du commerce dont les rétro-ingénieries sont maitrisées.
Ce démonstrateur permettra notamment d’interfacer votre travail avec celui des GLRT. Pour cette partie il sera
nécessaire de coder en C++. \\
Toute l'équipe
\newpage
\section{Specifications}
\subsection{Nouvelle norme}
Proposer une nouvelle norme de communications entre capteurs et box.
\subsection{Différenciation des appareils électriques}
Capacité à différencier des catégories d'appareils électriques.
\subsection{Architecture du projet}
A la suite de plusieurs brainstorming, nous souhaitons vous proposer une architecture différente de celle présente actuellement.
\subsubsection{Architecture initiale}
Pour rappel, voici le mode de fonctionnement initialement envisagé.
\begin{itemize}
  \item \textbf{Un serveur signal} :
  \begin{itemize}
    \item récupère les trames des capteurs, les converties en données utilisables (30°C sur le capteur \#2, 240 W sur le capteur \#3...).
    \item interprètes les trames : recherche d'allumage d'équipement, d'arrêt d'équipement.
    \item envoie les données dans une base de données (BDD) de :
    \begin{itemize}
      \item Lors de la réception d'une donnée capteur.
      \item Lors de la détection d'un évenement (allumage, arrêt équipement).
    \end{itemize}
  \end{itemize}
  \item \textbf{BDD} : stocke les évenements ainsi que les données capteurs
  \item \textbf{Un serveur Web} : gère l'affichage des informations de la BDD, ainsi que les erreurs commises par la partie signal sur les évenements. Doit être capable de traiter des faux positifs de la partie signal. Exemple : erreur réalisé par la partie signal : une lampe et un PC qui s'allume ont le même ID, nécessité de séparation de ces élements.
\end{itemize}
\subsubsection{Architecture Proposée}
Voici notre nouvelle proposition avec 4 serveurs :
\begin{itemize}
  \item \textbf{Un serveur signal récepteur des données des capteurs"} :\\
  permet la conversion des trames reçues en données à mettre dans la BDD. Schéma proposé : (id\_capteur,type\_donnée,valeur). Envoi des trames à la BDD et au serveur interprète, décrit ci-dessous, uniquement lorsqu'une donnée des capteurs est modifiée. Ce serveur permettra la génération des graphes de consommation par équipement (à la demande du serveur Web).
  \item \textbf{Un serveur signal interprète des données capteurs} :
  \begin{itemize}
    \item  activé à la réception d'information du serveur signal récepteur des données des capteurs".
    \item Réalise toute la partie probabilité, et lit des informations dans la BDD si nécessaire.
    \item Si un évenement allumage, arrêt, mauvais fonctionnement d'un équipement est valide alors ce serveur envoit une trame du type : (id\_event,timestamp,value) à la BDD.
  \end{itemize}
  \item \textbf{la BDD} : \\
  reçoit les trames des serveurs signal et permet au serveur info Web de traiter les données.
  \item \textbf{Un serveur info Web} :\\
  Permet de signaler une erreur sur un périphérique et de faire remonter l'information au serveur interprète des données des capteurs. Il permet notamment d'envoyer les données au client web pour l'affichage des graphes.
  \item \textbf{Le client info Web} :\\
  Affiche tous les graphes et les notifications essentielles pour l'utilisateur.
\end{itemize}
\subsubsection{Bilan}
\textbf{Avantages} nouvelle proposition sur originale :
\begin{itemize}
  \item Le code est plus optimisé pour la partie signal. Concernant la partie informatique, les technologies seront plus légères et permettront une meilleure gestion des accès à la BDD ainsi qu'une meilleure gestion des compétences (évite de l'interprétation sur l'allumage/l'arrét d'équipement dans la partie Web).
  \item Evite de séparer l'intépretation sur la partie Web et sur la partie signal, toute l'interprétation se fait sur le serveur interprète.
  \item Les informations sont stockées uniquement dans la BDD, pas de nécessité de doublons dans les serveurs de la partie signal.
  \item Meilleure fiabilité dans la gestion des trames : évite des temps de traitement trop long dans la partie signal à la réception des trames.
\end{itemize}
\textbf{Inconvénients}
\begin{itemize}
  \item Il est nécessaire que la solution proposée apportant trois serveurs différents ne consomme pas plus que l'énergie qui peut être économisée.
\end{itemize}

\subsection{Mise en place des serveurs}

Création de 3 serveurs et de la base de données:

\begin{itemize}
  \item La base de données :
  \begin{itemize}
    \item Déterminer la nature des objets stockés.
    \item Stocker uniquement les modifications de valeurs => garantie par le serveur signal de récupération
    \item Utilisation d'une technologie légère.
    \item Accès en lecture seule par le serveur signal interprète et le client Web. Ecriture réalisée par le serveur Web uniquement.
  \end{itemize}
\end{itemize}

\begin{itemize}
  \item Serveur Web :
  \begin{itemize}
    \item Validation des capteurs détectés par l'utilisateur.
    \item Validation par l'utilisateur des appareils mis sous tension dans le cas d'ambiguïté.
    \item Enregistrement des données reçues dans une base de données.
    \item L'utilisateur doit pouvoir corriger une fausse affectation.
    \item Différents affichages pour l'utilisateur :
    \begin{itemize}
      \item Bilan de consommation électrique sur un jour, une semaine, un mois, un an avec une étiquette d'allumage et d'arrêt des périphériques électriques.
      \item Vue représentant la consommation électrique d'un appareil.
      \item Vue représentant la consommation globale, suivi des consommations de tous les périphériques. Possibilité de restreindre cette vue aux courbes de quelques périphériques.
      \item Vue représentant l'impact de chaque appareil sur la consommation électrique pendant une durée de temps finie.
      \item Vue représentant le temps de mise en marche de chaque appareil pendant une durée de temps finie.
      \item Affichage des courbes de température, d'hygrométrie, de consommation d'eau et de consommation électrique.
    \end{itemize}
  \end{itemize}
\end{itemize}

\begin{itemize}
  \item Serveur signal récepteur des trames :
  \begin{itemize}
    \item Décodage trame.
    \item Détermination la structure de l'objet envoyé. Détection de modification sur la valeur d'un capteur.
    \item Information envoyée type : (Az23as,T,30) (Id\_capteur,Type d'info (T: température,P : puissance...),Valeur).
    \item Output en terminal des trames reçues / envoyées.
  \end{itemize}
\end{itemize}

\begin{itemize}
\item Serveur interprète
\begin{itemize}
  \item Capacité d'envoi de trame vers le serveur Web en cas de doute de détection entre plusieurs périphériques, ou lors d'un manque d'information par rapport un périphérique.
  \item Les informations utilisées par le serveur sont toutes stockées dans la BDD.
  \item Calcul de la probabilité d'événement répété à chaque modification d'une valeur d'un des capteurs.
\end{itemize}
\end{itemize}

\section{Proposition de technologie}
\begin{itemize}
\item Parties signals en C++.
\item Serveurs web codés en PHP ou .NET1.
\item Partie client en HTML, CSS, Javascript (AngularJS, Bootstrap, Foundation).
\end{itemize}

\section{Exemple : Scénario}
\begin{itemize}
  \item Dans une maison de capteurs, le capteur a2sI1q détecte 120W de consommation.
  \item Le serveur signal reçoit ensuite une différence de + 60 W par rapport à la dernière trame \"sauvegardée" sur le serveur signal (a2sI1q;P;120).
  \item Le serveur signal envoie au serveur web la trame qui sera stockée dans la BDD info. De même, le serveur interprète sera actif et pourra vérifier son interprétation avec la BDD info.
  \item Suite à l'interprétation (ex: Mise en route du frigo), un message est envoyé au serveur web qui sera ensuite affiché sur l'interface utilisateur.(id\_event,timestamp,ON)
  \item Enfin, le serveur web met à jour les courbes de consommation.
\end{itemize}





\section{Planning}
\subsection{Recherche elec}
\begin{itemize}
	\item Présentation des résultats des recherches bibliographique : Lundi 27 octobre
	\item Présentation du premier algorithme : Lundi 3 novembre
\end{itemize}
\subsection{Recherche couche physique}
\begin{itemize}
	\item Présentation des recherches : Lundi 20 octobre
\end{itemize}
\subsection{Serveur Signal}
Lundi 20 octobre
\subsection{Serveur Interprétation}
Décembre
\subsection{Serveur web et BDD}
Première ébauche: Vendredi 31 octobre BDD 
\subsection{Interface Client Web}
Maquette: Vendredi 24 octobre 
\end{document}


%\begin{figure}[!h]
%\begin{center}
%\includegraphics[scale = 0.4]{./fig/fig1.jpg}
%\caption{figure}
%\label{Figure}
%\end{center}
%\end{figure}
