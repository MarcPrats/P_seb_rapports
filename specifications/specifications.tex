\documentclass[10pt,a4paper]{article}
\usepackage[utf8]{inputenc}
\usepackage[T1]{fontenc}
\usepackage[french]{babel}
\usepackage{enumitem}
%\frenchbsetup{StandardLists=true}
\usepackage{caption}
\usepackage{graphicx}
\usepackage{amsmath}
\usepackage{subfigure}
\usepackage{appendix}
\usepackage{listings}
\usepackage{multicol}
\usepackage{moreverb}
\usepackage{epsfig}
\usepackage{textcomp}


    \usepackage{xcolor}
      %%%%%%%%%%%%%%%%  inclure la source %%%%%%%%%%%%%%%%%%%%
  \newcommand*\styleC{\fontsize{9}{10pt}\usefont{T1}{cmtt}{m}{n}\selectfont }
  \newcommand*\styleD{\fontsize{9}{10pt}\usefont{T1}{cmtt}{m}{n}\selectfont }

      \makeatletter
      % on fixe le langage utilisé
      \lstset{language=matlab, breaklines=true}
      \edef\Motscle{emph={\lst@keywords}}
      \expandafter\lstset\expandafter{%
       \Motscle}
      \makeatother


      \definecolor{Vvert}{rgb}{0.133,0.545,0.133}

      \lstset{emphstyle=\color{blue}, % les mots réservés de matlab en bleu
      basicstyle=\styleC,
      keywordstyle=\cmtt,
      commentstyle=\color{Vvert}\styleD, % commentaire en gris
      numberstyle=\tiny\color{red},
      numbers=left,
      numbersep=10pt,
      lineskip=0.7pt,
      showstringspaces=false}
      %  % inclure le fichier source
      \newcommand{\FSource}[1]{%
      \lstinputlisting[texcl=true]{#1}
      }

\setlength{\textwidth}{16cm} % Largeur du texte
\setlength{\textheight}{24cm} % Hauteur du texte
\setlength{\evensidemargin}{-0.2cm} % Taille des marges pour les pages paires
\setlength{\oddsidemargin}{-0.2cm} % Taille des marges pour les pages impaires
\setlength{\topmargin}{-1.5cm} % Taille de l'ent\^ete

\title{PR307 Smart-EcoBox : \\
Spécifications}
\author{LETURC Xavier <xleturc@enseirb-matmeca.fr> \\
DE CACQUERAY VALMENIER Mathias <mdecacquerayvalmenie@enseirb-matmeca.fr> \\
BOUCHAIN Antoine <abouchain@enseirb-matmeca.fr> \\
RAMBAUD Benjamin <brambaud@enseirb-matmeca.fr> \\
PRATS ESCRIBANO Marc <mpratsescribano@enseirb-matmeca.fr> \\
TOUFIQUE Karim <ktoufique@enseirb-matmeca.fr> \\
GOZLAN Melinda <mgozlan@enseirb-matmeca.fr> }


\begin{document}

\maketitle

\begin{figure}[ht]
\begin{center}
%\noindent \includegraphics[scale = 0.3]{../../../Logo_ENSEIRB.pdf}
\end{center}
\end{figure}


\newpage

\tableofcontents

\newpage

\section{Organisation}
Mathias de Cacqueray-Valmenier est chef de projet, il est secondé pour la partie informatique par Marc Prats Escribano.
\subsection{Partie Informatique}
La partie informatique de ce projet comporte 3 axes :\\
\textbf{Axe 1 :} Trouver une solution technique concernant le stockage des données récupérées depuis les capteurs. Ces
données peuvent être volumineuses.
\begin{itemize}
\item JP
\end{itemize}
\textbf{Axe 2 :} Développer des vues qui permettent de visualiser les données stockées de façon intelligible.
\begin{itemize}
\item JP
\end{itemize}
\textbf{Axe 3 :} Permettre aux utilisateurs de configurer l’outil : changer les noms des appareils éléctriques, configurer la
façon dont les données sont affichées, ...
\begin{itemize}
\item JP
\end{itemize}
\subsection{Partie Signal}
La partie signal/communications numériques de ce projet comporte 3 axes :\\
\textbf{Axe 1 :} Etude bibliographique et implémentations matlab des méthodes/algorithmes permettant l’identification
d’appareils électriques via l’analyse des signaux qu’ils génèrent sur le réseau électrique lors de leur allumage,
\begin{itemize}
\item Antoine Bouchain
\item Xavier Leturn
\end{itemize}
\textbf{Axe 2 :} Propositions de couches physiques (box + capteurs) fiable, efficace et cohérente avec des communications
dans la bande ISM 433MHz. La faisabilité de ces propostions devra être montrée par simulation Matlab.
\begin{itemize}
\item Mélinda Gozlan
\item Mathias de Cacqueray-Valmenier
\end{itemize}
\textbf{Axe 3 :} Réalisation d’un démonstrateur avec des capteurs du commerce dont les rétro-ingénieries sont maitrisées.
Ce démonstrateur permettra notamment d’interfacer votre travail avec celui des GLRT. Pour cette partie il sera
nécessaire de coder en C++. Noter que pour cet axe une partie du travail a été réalisé en projet S8 et elle vous sera
fourni.\\
Toute l'équipe
\end{document}


%\begin{figure}[!h]
%\begin{center}
%\includegraphics[scale = 0.4]{./fig/fig1.jpg}
%\caption{figure}
%\label{Figure}
%\end{center}
%\end{figure}
