\documentclass[10pt,a4paper]{article}
\usepackage[utf8]{inputenc}
\usepackage[T1]{fontenc}
\usepackage[french]{babel}
\usepackage{enumitem}
%\frenchbsetup{StandardLists=true}
\usepackage{caption}
\usepackage{graphicx}
\usepackage{amsmath}
\usepackage{subfigure}
\usepackage{appendix}
\usepackage{listings}
\usepackage{multicol}
\usepackage{moreverb}
\usepackage{epsfig}
\usepackage{textcomp}


    \usepackage{xcolor}
      %%%%%%%%%%%%%%%%  inclure la source %%%%%%%%%%%%%%%%%%%%
  \newcommand*\styleC{\fontsize{9}{10pt}\usefont{T1}{cmtt}{m}{n}\selectfont }
  \newcommand*\styleD{\fontsize{9}{10pt}\usefont{T1}{cmtt}{m}{n}\selectfont }

      \makeatletter
      % on fixe le langage utilisé
      \lstset{language=matlab, breaklines=true}
      \edef\Motscle{emph={\lst@keywords}}
      \expandafter\lstset\expandafter{%
       \Motscle}
      \makeatother


      \definecolor{Vvert}{rgb}{0.133,0.545,0.133}

      \lstset{emphstyle=\color{blue}, % les mots réservés de matlab en bleu
      basicstyle=\styleC,
      keywordstyle=\cmtt,
      commentstyle=\color{Vvert}\styleD, % commentaire en gris
      numberstyle=\tiny\color{red},
      numbers=left,
      numbersep=10pt,
      lineskip=0.7pt,
      showstringspaces=false}
      %  % inclure le fichier source
      \newcommand{\FSource}[1]{%
      \lstinputlisting[texcl=true]{#1}
      }

\setlength{\textwidth}{16cm} % Largeur du texte
\setlength{\textheight}{24cm} % Hauteur du texte
\setlength{\evensidemargin}{-0.2cm} % Taille des marges pour les pages paires
\setlength{\oddsidemargin}{-0.2cm} % Taille des marges pour les pages impaires
\setlength{\topmargin}{-1.5cm} % Taille de l'ent\^ete

\title{Cours PR307 : Formation Projet S9}
\author{Mathias de Cacqueray-Valmenier }


\begin{document}

\maketitle

\begin{figure}[ht]
\begin{center}
\noindent \includegraphics[scale = 0.3]{../../../Logo_ENSEIRB.pdf}
\end{center}
\end{figure}


\newpage

\tableofcontents

\newpage

\section{Planning projet}
\textbf{Liste de tâches}
\begin{itemize}
\item Début
\item Fin
\item Durée
\item Ressources
\item Couts
\item Séquencement
\item Jalons données d'entrée / données de sortie
\item La marge !
\item Métrique d'avancement /réalisation
\end{itemize}
\textbf{Revues régulières}
\begin{itemize}
\item au bon rythme
\item au bon niveau de détails
\end{itemize}
\textbf{Mise en place d'outils de gestion}
\begin{itemize}
\item Indicateurs
\item Métriques
\end{itemize}
Permettre d'avoir rapidement un état représentatif fiable de l'avancement.\\
Détecter au plus tôt les dérives de couts, de temps, d'activité.\\
\textbf{Le plus tôt possible, pour anticiper / réagir}
\section{Gérer les aléas et les risques, saisir les opportunités}
\begin{itemize}
\item Aléas : l'imprévu, il faut identifier de la marge.
\item Risques : prévisibles
\begin{itemize}
\item Identifier les risques lors de la phase offre
\item Mettre à jour la base des risques régulières.
\item Identifier les actions permettant de limiter l'occurrence des risques.
\item  Mettre en place un plan d'action pour limiter ces risques si le jeu en vaut la chandelle.
\item L'expérience est l'élément clé pour l'identification et la gestion des risques (experts, retour d'expérience...)
\end{itemize}
\item Opportunités : Il faut les saisir
\begin{itemize}
\item Comme les risques : identifier, MaJ, plan d'action
\item Chercher les conditions qui permettent de les saisir !
\end{itemize}
\end{itemize}
\begin{itemize}
\item Faire un avancement régulier à la bonne fréquence
\item Se garder de la marge pour réagir
\item Découper les taches en taches pilotables
\begin{itemize}
\item durée, ni trop courte, ni trop longue
\item nombre d'intervenants humainement contrôlable
\end{itemize}
\item Identifier des jalons répartis régulièrement
\item Mettre en place les plans d'actions au plus tôt
\begin{itemize}
\item Négo client
\item Plan B
\end{itemize}
\end{itemize}
\end{document}


%\begin{figure}[!h]
%\begin{center}
%\includegraphics[scale = 0.4]{./fig/fig1.jpg}
%\caption{figure}
%\label{Figure}
%\end{center}
%\end{figure}
