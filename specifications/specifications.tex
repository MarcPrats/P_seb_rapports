\documentclass[10pt,a4paper]{article}
\usepackage[utf8]{inputenc}
\usepackage[T1]{fontenc}
\usepackage[french]{babel}
\usepackage{enumitem}
%\frenchbsetup{StandardLists=true}
\usepackage{caption}
\usepackage{graphicx}
\usepackage{amsmath}
\usepackage{subfigure}
\usepackage{appendix}
\usepackage{listings}
\usepackage{multicol}
\usepackage{moreverb}
\usepackage{epsfig}
\usepackage{textcomp}


    \usepackage{xcolor}
      %%%%%%%%%%%%%%%%  inclure la source %%%%%%%%%%%%%%%%%%%%
  \newcommand*\styleC{\fontsize{9}{10pt}\usefont{T1}{cmtt}{m}{n}\selectfont }
  \newcommand*\styleD{\fontsize{9}{10pt}\usefont{T1}{cmtt}{m}{n}\selectfont }

      \makeatletter
      % on fixe le langage utilisé
      \lstset{language=matlab, breaklines=true}
      \edef\Motscle{emph={\lst@keywords}}
      \expandafter\lstset\expandafter{%
       \Motscle}
      \makeatother


      \definecolor{Vvert}{rgb}{0.133,0.545,0.133}

      \lstset{emphstyle=\color{blue}, % les mots réservés de matlab en bleu
      basicstyle=\styleC,
      keywordstyle=\cmtt,
      commentstyle=\color{Vvert}\styleD, % commentaire en gris
      numberstyle=\tiny\color{red},
      numbers=left,
      numbersep=10pt,
      lineskip=0.7pt,
      showstringspaces=false}
      %  % inclure le fichier source
      \newcommand{\FSource}[1]{%
      \lstinputlisting[texcl=true]{#1}
      }

\setlength{\textwidth}{16cm} % Largeur du texte
\setlength{\textheight}{24cm} % Hauteur du texte
\setlength{\evensidemargin}{-0.2cm} % Taille des marges pour les pages paires
\setlength{\oddsidemargin}{-0.2cm} % Taille des marges pour les pages impaires
\setlength{\topmargin}{-1.5cm} % Taille de l'ent\^ete

\title{PR307 Smart-EcoBox : \\
Spécifications}
\author{
BOUCHAIN Antoine <abouchain@enseirb-matmeca.fr> \\
DE CACQUERAY VALMENIER Mathias <mdecacquerayvalmenie@enseirb-matmeca.fr> \\
GOZLAN Melinda <mgozlan@enseirb-matmeca.fr>\\
LETURC Xavier <xleturc@enseirb-matmeca.fr> \\
PRATS ESCRIBANO Marc <mpratsescribano@enseirb-matmeca.fr> \\
RAMBAUD Benjamin <brambaud@enseirb-matmeca.fr> \\
TOUFIQUE Karim <ktoufique@enseirb-matmeca.fr> \\ }


\begin{document}
\maketitle

\begin{figure}[ht]
\begin{center}
\noindent \includegraphics[scale = 0.25]{../../Logo_ENSEIRB.pdf}
\end{center}
\end{figure}
\begin{figure}[ht]
\begin{center}
\noindent \includegraphics[scale = 0.25]{../../smartee_logo.pdf}
\end{center}
\end{figure}


\newpage

\tableofcontents

\newpage

\section{Organisation}
Mathias de Cacqueray-Valmenier est chef de projet, il est secondé pour la partie informatique par Marc Prats Escribano.
\subsection{Partie Informatique}
La partie informatique de ce projet comporte 3 axes :\\
\textbf{Axe 1 :} Trouver une solution technique concernant le stockage des données récupérées depuis les capteurs. Ces
données peuvent être volumineuses.
\begin{itemize}
\item Karim Toufik
\item Benjamin Rambaud
\end{itemize}
\textbf{Axe 2 :} Développement site Web
\begin{itemize}
\item Marc Prats Escribano
\end{itemize}
\subsection{Partie Signal}
La partie signal/communications numériques de ce projet comporte 3 axes :\\
\textbf{Axe 1 :} Etude bibliographique et implémentations matlab des méthodes/algorithmes permettant l’identification
d’appareils électriques via l’analyse des signaux qu’ils génèrent sur le réseau électrique lors de leur allumage,
\begin{itemize}
\item Antoine Bouchain
\item Xavier Leturc
\end{itemize}
\textbf{Axe 2 :} Propositions de couches physiques (box + capteurs) fiable, efficace et cohérente avec des communications
dans la bande ISM 433MHz. La faisabilité de ces propostions devra être montrée par simulation Matlab.
\begin{itemize}
\item Mélinda Gozlan
\item Mathias de Cacqueray-Valmenier
\end{itemize}
\textbf{Axe 3 :} Réalisation d’un démonstrateur avec des capteurs du commerce dont les rétro-ingénieries sont maitrisées.
Ce démonstrateur permettra notamment d’interfacer votre travail avec celui des GLRT. Pour cette partie il sera
nécessaire de coder en C++. \\
Toute l'équipe
\newpage
\section{Specifications}
\subsection{Nouvelle norme}
Proposer une nouvelle norme de communications entre capteurs et box.
\subsection{Différenciation des appareils électriques}
Capacité à différencier des catégories d'appareils électriques.
\subsection{Architecture du projet}
A la suite de plusieurs brainstorming, nous souhaitons vous proposer une architecture différente de celle présente actuellement.
\subsubsection{Architecture initiale}
Pour rappel, voici le mode de fonctionnement initialement envisagé.
\begin{itemize}
  \item \textbf{Un serveur signal} :
  \begin{itemize}
    \item Récupère les trames des capteurs, les converties en données utilisables (30°C sur le capteur \#2, 240 W sur le capteur \#3...).
    \item Interprète les trames : recherche d'allumage d'équipement, d'arrêt d'équipement.
    \item Envoie les données dans une base de données (BDD) :
    \begin{itemize}
      \item Lors de la réception d'une donnée capteur.
      \item Lors de la détection d'un évènement (allumage, arrêt équipement).
    \end{itemize}
  \end{itemize}
  \item \textbf{BDD} : stocke les évenements ainsi que les données capteurs
  \item \textbf{Un serveur Web} : gère l'affichage des informations de la BDD, ainsi que les erreurs commises par la partie signal sur les évenements. Doit être capable de traiter des faux positifs de la partie signal. Exemple : erreurs réalisées par la partie signal : une lampe et un PC qui s'allument ont le même ID, nécessité de séparation de ces élements.
\end{itemize}
\subsubsection{Architecture Proposée}
Voici notre nouvelle proposition avec 4 serveurs :
\begin{itemize}
  \item \textbf{Un serveur signal récepteur des données des capteurs} :\\
  permet la conversion des trames reçues en données à mettre dans la BDD. Schéma proposé : (id\_capteur, timestamp, type\_donnée, valeur). Envoi des trames à la BDD et au serveur interprète, décrit ci-dessous, uniquement lorsqu'une donnée des capteurs est modifiée. Ce serveur permettra la génération des graphes de consommation par équipement (à la demande du serveur Web).
  \item \textbf{Un serveur signal interprète des données capteurs} :
  \begin{itemize}
    \item Activé à la réception d'information du \textit{serveur signal récepteur des données des capteurs}.
    \item Calcule les probabilités d'évènement, et lit des informations dans la BDD si nécessaire.
    \item Si un évènement est valide alors ce serveur envoit une trame du type : (id\_appareil, timestamp, value) à la BDD.
  \end{itemize}
  \item \textbf{la BDD} : \\
  reçoit les informations des serveurs signal et permet au serveur info Web de traiter les données.
  \item \textbf{Un serveur info Web} :\\
  Permet de faire remonter une erreur repérée par l'utilisateur au serveur interprète des données des capteurs. Il permet notamment d'envoyer les données au client web pour l'affichage des graphes.
  \item \textbf{Le client info Web} :\\
  Affiche tous les graphes et les notifications essentielles pour l'utilisateur. Permet à l'utilisateur de signaler une erreur lié au système et de le configurer certains paramètres.
\end{itemize}
\subsubsection{Bilan}
\textbf{Avantages} nouvelle proposition sur originale :
\begin{itemize}
  \item Le code est plus optimisé pour la partie signal. Concernant la partie informatique, les technologies sont plus légères et permettent une meilleure gestion des accès à la BDD et évite des calculs d'interprétation sur la partie Web.
  \item Evite de séparer l'intépretation sur la partie Web et sur la partie signal, toute l'interprétation se fait sur le serveur dédié.
  \item Les informations sont stockées uniquement dans la BDD, pas de nécessité de doublons dans les serveurs de la partie signal.
  \item Meilleure fiabilité dans la gestion des trames : évite des temps de traitement trop long dans la partie signal à la réception des trames.
\end{itemize}
\textbf{Inconvénients}
\begin{itemize}
  \item Il est nécessaire que la solution proposée apportant trois serveurs différents ne consomme pas plus que l'énergie qui peut être économisée.
\end{itemize}

\subsection{Mise en place des serveurs}

Création de 3 serveurs et de la base de données:

\begin{itemize}
  \item \textbf{La base de données} :
  \begin{itemize}
    \item Stocke uniquement les modifications de valeurs.
    \item Utilise une technologie légère.
    \item Accès en lecture seule par le serveur signal interprète et le client Web. Écriture réalisée par le serveur Web uniquement.
  \end{itemize}
\end{itemize}

\begin{itemize}
  \item \textbf{Serveur Web / Client Web} :
  \begin{itemize}
    \item Validation des capteurs détectés par l'utilisateur.
    \item Validation par l'utilisateur des appareils mis sous tension dans le cas d'ambiguïté.
    \item Enregistrement des données reçues dans une base de données.
    \item L'utilisateur doit pouvoir corriger une fausse affectation.
    \item Différents affichages pour l'utilisateur :
    \begin{itemize}
      \item Bilan de consommation électrique sur un jour, une semaine, un mois, un an avec une étiquette d'allumage et d'arrêt des périphériques électriques.
      \item Vue représentant la consommation électrique d'un appareil.
      \item Vue représentant la consommation globale, suivi des consommations de tous les périphériques. Possibilité de restreindre cette vue aux courbes de quelques périphériques.
      \item Vue représentant l'impact de chaque appareil sur la consommation électrique pendant une durée de temps finie.
      \item Vue représentant le temps de mise en marche de chaque appareil pendant une durée de temps finie.
      \item Affichage des courbes de température, d'hygrométrie, de consommation d'eau et de consommation électrique.
    \end{itemize}
  \end{itemize}
\end{itemize}

\begin{itemize}
  \item \textbf{Serveur signal récepteur des trames} :
  \begin{itemize}
    \item Décodage trame.
    \item Vérification de modifications sur la valeur d'un capteur.
    \item Information envoyée type : (Az23as,T,30,timestamp) (Id\_capteur,Type d'info (T: température,P : puissance...),Valeur).
    \item Affichage d'un un terminal des trames reçues / envoyées.
  \end{itemize}
\end{itemize}

\begin{itemize}
\item \textbf{Serveur interprète}
\begin{itemize}
  \item Capacité d'envoi d'évènement vers le serveur Web en cas de doute de détection entre plusieurs périphériques, ou lors d'un manque d'information vis à vis d'un périphérique.
  \item Les informations que lit le serveur interprète sont stockés dans la BDD.
  \item Calcul de la probabilité d'événement répété à chaque modification d'une valeur d'un des capteurs.
\end{itemize}
\end{itemize}

\section{Propositions de technologie}
\begin{itemize}
\item Parties signal en C++.
\item Serveurs web codés en PHP ou J2EE.
\item Partie client en HTML, CSS, Javascript (AngularJS, Bootstrap, Foundation).
\end{itemize}

\newpage
\section{Exemple : Scénario}

On vient de tout brancher, la box s'allume (récupération sans-fils):
\begin{enumerate}
	\item Lecture de la BDD Noms : \fbox{id\_capteurs, nom\_capteurs}
	\item Fork (récupération USB)
	\item Fork (interprétation) 
\end{enumerate}
\begin{enumerate}
	\item Récupération sans-fils : Récupère la trame T°, Débit ou Consommation électrique des capteurs sans fil.
	\begin{itemize}
		\item Si \textbf{ID connu} \fbox{id\_cpt1, nom\_cpt1, 20°C} 
		\begin{itemize}
			\item Envoi de la trame à la BDD Data \fbox{id\_cpt1,TimeStamp,valeur} 	
		\end{itemize} 
	\end{itemize}
	\begin{itemize}
		\item Si \textbf{ID inconnu}
		\begin{itemize}
			\item Création nom "cpt2"
			\item Envoi de la trame à la BDD Nom \fbox{id\_cpt2, nom\_cpt2\_default}
			\item Envoi de la trame à la BDD Data \fbox{id\_cpt2,TimeStamp,val}
		\end{itemize}
	\end{itemize}
	\item Récupération USB : 
	\begin{itemize}
		\item Trame Puissance sous échantillonné pour écrire dans la BDD  \textit{Événements Sauvegardés}
	\end{itemize}
	\item Interprétation : 
	\begin{itemize}
		\item Reçoit donné puissance brute de \textit{récupération USB}
	\end{itemize}	 
	\item Algorithmes d'interprétation : détection d'un événement
	\begin{itemize}		
		\item Lecture BDD  \textit{Événements Sauvegardés} pour recherche correspondance.
		\item Si \textbf{Pas de doute} sur l'événement
		\begin{itemize}
			\item Ecriture BDD event "Allumage R1, P1" : \fbox{id\_evt,TimeStamp,valeur=allumge}
		\end{itemize} 
		\item Si \textbf{Doute} sur l'événement
		\begin{itemize}
			\item Lecture BDD 
			\item Si la BDD ne permet pas d'aider à différencier \\
			-> proposer un éventail possible d'appareils
			\item proposer les 3 possibilités les plus probables.
			\item Ecriture BDD doute \fbox{id\_evt\_doute,P1,id doute 1, P2, id doute 2, P3,id doute 3}
			\item Sauvegarde des points d'intéret de l'événement non reconnu dans la BDD  \textit{Événements Sauvegardés} : \fbox{id\_evt\_doute,$POI_1$,$POI_2$,$POI_3$,$P0I_4$,$POI_5$}
		\end{itemize} 
	\end{itemize}
\end{enumerate}

\newpage
\section{Formalisation des tables BDD}
\textbf{Nom} : fait uniquement un rapprochement entre l'id unique d'un capteur, d'un événement ou d'un événement en doute et le nom associé.
\begin{table}[!h]
\begin{tabular}{|l|l|l|}
\hline
id\_capteurs,id\_evt, id\_evt\_doute & localistion Capteur/Nom Matériel & Type\_data(T,P,D,evt) \\ \hline
 &  &  \\
 &  & 
\end{tabular}
\end{table}\\
\textbf{Data Capteurs / Événements} : Enregistre les données des capteurs et des événements ou il n'y pas de doute.
\begin{table}[!h]
\begin{tabular}{|l|l|l|}
\hline
id\_capteurs,id\_evt & TimeStamp & Valeur \\ \hline
 &  &  \\
 &  & 
\end{tabular}
\end{table}\\
\textbf{Doute} : Dans le cas d'un doute on donne les probabilités des trois événements les plus probable. Un nouvelle id est associé a cette événement.
\begin{table}[!h]
\begin{tabular}{|l|l|l|l|l|l|l|}
\hline
id\_evt\_doute & $id_1$ & $P_1$ & $id_2$ & $P_2$ & $id_3$ & $P_3$  \\ \hline
 &  & &  & &  &  \\
 &  &  &  & &  &
\end{tabular}
\end{table}\\
\textbf{Événements Sauvegardés} : Base de données des connaissances des points d'intèrets pour interprétation, ajout de données dans le cas d'un événement en doute.
\begin{table}[!h]
\begin{tabular}{|l|l|l|l|l|l|}
\hline
id\_evt & $POI_1$ & $POI_2$ & $POI_3$ & $P0I_4$ & $POI_5$   \\ \hline
 &  & &  & &  \\
 &  &  &  & &
\end{tabular}
\end{table}


\newpage
\section{Planning}
\subsection{Recherche elec}
\begin{itemize}
	\item Présentation des résultats des recherches bibliographique : Lundi 27 octobre
	\item Présentation du premier algorithme : Lundi 3 novembre
\end{itemize}
\subsection{Recherche couche physique}
\begin{itemize}
	\item Présentation des recherches : Lundi 20 octobre
\end{itemize}
\subsection{Serveur Récupération}
Lundi 20 octobre
\subsection{Serveur Interprétation}
Décembre
\subsection{Serveur web et BDD}
Première ébauche: Vendredi 31 octobre BDD
\subsection{Interface Client Web}
Maquette: Vendredi 24 octobre
\end{document}


%\begin{figure}[!h]
%\begin{center}
%\includegraphics[scale = 0.4]{./fig/fig1.jpg}
%\caption{figure}
%\label{Figure}
%\end{center}
%\end{figure}
