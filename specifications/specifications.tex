\documentclass[10pt,a4paper]{article}
\usepackage[utf8]{inputenc}
\usepackage[T1]{fontenc}
\usepackage[french]{babel}
\usepackage{enumitem}
%\frenchbsetup{StandardLists=true}
\usepackage{caption}
\usepackage{graphicx}
\usepackage{amsmath}
\usepackage{subfigure}
\usepackage{appendix}
\usepackage{listings}
\usepackage{multicol}
\usepackage{moreverb}
\usepackage{epsfig}
\usepackage{textcomp}


    \usepackage{xcolor}
      %%%%%%%%%%%%%%%%  inclure la source %%%%%%%%%%%%%%%%%%%%
  \newcommand*\styleC{\fontsize{9}{10pt}\usefont{T1}{cmtt}{m}{n}\selectfont }
  \newcommand*\styleD{\fontsize{9}{10pt}\usefont{T1}{cmtt}{m}{n}\selectfont }

      \makeatletter
      % on fixe le langage utilisé
      \lstset{language=matlab, breaklines=true}
      \edef\Motscle{emph={\lst@keywords}}
      \expandafter\lstset\expandafter{%
       \Motscle}
      \makeatother


      \definecolor{Vvert}{rgb}{0.133,0.545,0.133}

      \lstset{emphstyle=\color{blue}, % les mots réservés de matlab en bleu
      basicstyle=\styleC,
      keywordstyle=\cmtt,
      commentstyle=\color{Vvert}\styleD, % commentaire en gris
      numberstyle=\tiny\color{red},
      numbers=left,
      numbersep=10pt,
      lineskip=0.7pt,
      showstringspaces=false}
      %  % inclure le fichier source
      \newcommand{\FSource}[1]{%
      \lstinputlisting[texcl=true]{#1}
      }

\setlength{\textwidth}{16cm} % Largeur du texte
\setlength{\textheight}{24cm} % Hauteur du texte
\setlength{\evensidemargin}{-0.2cm} % Taille des marges pour les pages paires
\setlength{\oddsidemargin}{-0.2cm} % Taille des marges pour les pages impaires
\setlength{\topmargin}{-1.5cm} % Taille de l'ent\^ete

\title{PR307 Smart-EcoBox : \\
Spécifications}
\author{LETURC Xavier <xleturc@enseirb-matmeca.fr> \\
DE CACQUERAY VALMENIER Mathias <mdecacquerayvalmenie@enseirb-matmeca.fr> \\
BOUCHAIN Antoine <abouchain@enseirb-matmeca.fr> \\
RAMBAUD Benjamin <brambaud@enseirb-matmeca.fr> \\
PRATS ESCRIBANO Marc <mpratsescribano@enseirb-matmeca.fr> \\
TOUFIQUE Karim <ktoufique@enseirb-matmeca.fr> \\
GOZLAN Melinda <mgozlan@enseirb-matmeca.fr> }


\begin{document}

\maketitle

\begin{figure}[ht]
\begin{center}
%\noindent \includegraphics[scale = 0.3]{../../../Logo_ENSEIRB.pdf}
\end{center}
\end{figure}


\newpage

\tableofcontents

\newpage

\section{Organisation}
Mathias de Cacqueray-Valmenier est chef de projet, il est secondé pour la partie informatique par Marc Prats Escribano.
\subsection{Partie Informatique}
La partie informatique de ce projet comporte 3 axes :\\
\textbf{Axe 1 :} Trouver une solution technique concernant le stockage des données récupérées depuis les capteurs. Ces
données peuvent être volumineuses.
\begin{itemize}
\item JP
\end{itemize}
\textbf{Axe 2 :} Développer des vues qui permettent de visualiser les données stockées de façon intelligible.
\begin{itemize}
\item JP
\end{itemize}
\textbf{Axe 3 :} Permettre aux utilisateurs de configurer l’outil : changer les noms des appareils éléctriques, configurer la
façon dont les données sont affichées, ...
\begin{itemize}
\item JP
\end{itemize}
\subsection{Partie Signal}
La partie signal/communications numériques de ce projet comporte 3 axes :\\
\textbf{Axe 1 :} Etude bibliographique et implémentations matlab des méthodes/algorithmes permettant l’identification
d’appareils électriques via l’analyse des signaux qu’ils génèrent sur le réseau électrique lors de leur allumage,
\begin{itemize}
\item Antoine Bouchain
\item Xavier Leturc
\end{itemize}
\textbf{Axe 2 :} Propositions de couches physiques (box + capteurs) fiable, efficace et cohérente avec des communications
dans la bande ISM 433MHz. La faisabilité de ces propostions devra être montrée par simulation Matlab.
\begin{itemize}
\item Mélinda Gozlan
\item Mathias de Cacqueray-Valmenier
\end{itemize}
\textbf{Axe 3 :} Réalisation d’un démonstrateur avec des capteurs du commerce dont les rétro-ingénieries sont maitrisées.
Ce démonstrateur permettra notamment d’interfacer votre travail avec celui des GLRT. Pour cette partie il sera
nécessaire de coder en C++. \\
Toute l'équipe
\newpage
\section{Specifications}
Réalisation :
\subsection{Nouvelle norme}
Proposer une nouvelle norme de communications entre capteurs et box.
\subsection{Différenciation des appareils électriques}
Capacité à différencier des catégories d'appareils électriques.

\subsection{Box Fonctionnelle}
\begin{itemize}
  \item Période d'initialisation avec enregistrement des capteurs à écouter.
  \item Écoute des capteurs enregistrés et envoi de chaque mesure (température, hygrométrie, consommation électrique, consommation d'eau) à la partie info + détection de mise sous tension ou arrêt d'appareil électrique.
  \item Stabilité des algorithmes de la Box.
\end{itemize}

\subsection{Mise en place des serveurs}

Création de 4 serveurs virtuels:

\begin{itemize}
\item Serveur base de donnée
\begin{itemize}
\item Qu'est-ce qu'on stocke ? (Stocker les fronts montants
\item Comment? 
\item Pourquoi?
\end{itemize} léger (SQLlite)
\end{itemize}

\begin{itemize}
\item Serveur Web/Info
\begin{itemize}
\item affichage seulement: Somme sur les étiquettes ON/OFF des appareils et/ou vue restreinte de quelques périphéries
\item Correction user: différentes échelles de temps
\end{itemize}
\end{itemize}

\begin{itemize}
\item Serveur signal
\begin{itemize}
\item Décodage trame
\item Qu'est-ce qu'on envoie? Détection dernière occurrence (ID capteurs)
\item Output en terminal	
\end{itemize}
\end{itemize}

\begin{itemize}
\item Serveur interprétation
\end{itemize}

\section{Informatique}
Réalisation :
\subsection{Box Fonctionnelle}
\begin{itemize}
  \item Validation des capteurs détectés par l'utilisateur.
  \item Validation par l'utilisateur des appareils mis sous tension dans le cas d'ambiguïté.
  \item Enregistrement des données reçues dans une base de donnée.
  \item L'utilisateur est censé pouvoir corriger une fausse affectation.
  \item Différents affichages pour l'utilisateur :
  \begin{itemize}
    \item Bilan de consommation électrique sur un jour, une semaine, un mois, un an avec étiquette d'allumage et d'arrêt des périphériques électriques.
    \item Vue avec la consommation d'un appareil.
    \item Vue avec la consommation globale, suivi des consommations de tout les périphériques. Possibilité de restreindre cette vue aux courbes de quelques périphériques.
    \item Vue avec l'impact de chaque appareil sur la consommation électrique par jour, semaine, mois ,an.
    \item Vue avec le temps d'allumage de chaque appareil par jour, semaine, mois ,an.
    \item Affichage des courbes température, hygrométrie, consommation d'eau, consommation électrique.
  \end{itemize}
\end{itemize}
\subsection{Proposition de technologie}
\begin{itemize}
\item Partie signal en C++
\item Partie serveur info en PHP et Node.js
\item Partie client en HTML, CSS, Javascript (AngularJS, Bootstrap, Foundation)
\end{itemize}

\section{Scénario}
Dans une maison de capteurs, le capteur PXX_123 détecte 120W de consommation.
Le serveur signal reçoit ensuite une différence de + 60 W par rapport à la dernière trame \"sauvegardées" sur le serveur signal
(PXX_123;P;120).
Le serveur signal envoie au serveur info la trame qui sera stockée dans la BDD info. De même, le serveur interprétation sera actif et pourra vérifier son interprétation avec la BDD info.
Suite à l'interprétation (ex: Mise en route du frigo), un message est envoyé sur l'interface utilisateur.(id_event,timestamp,ON)
Serveur info met à jour les courbes de consommation.

\section{Planning}
\subsection{Recherche elec}
\begin{itemize}
	\item Présentation des résultats des recherches bibliographique : Lundi 27 octobre
	\item Présentation du premier algo : Lundi 3 novembre
\end{itemize}
\subsection{Recherche couche phy}
\begin{itemize}
	\item Présentation Recherches : Lundi 20 octobre
\end{itemize}
\subsection{Serveur Signal}
Lundi 20 octobre
\subsection{Serveur Interprétation}
Décembre
\subsection{Serveur BDD}
Vendredi 31 BDD ébauche
\subsection{Serveur Web}
Vendredi 24 maquette
\end{document}


%\begin{figure}[!h]
%\begin{center}
%\includegraphics[scale = 0.4]{./fig/fig1.jpg}
%\caption{figure}
%\label{Figure}
%\end{center}
%\end{figure}
